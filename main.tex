%% This is file `DEMO-TUDaThesis.tex' version 3.07 (2020/10/21),
%% it is part of
%% TUDa-CI -- Corporate Design for TU Darmstadt
%% ----------------------------------------------------------------------------
%%
%%  Copyright (C) 2018--2020 by Marei Peischl <marei@peitex.de>
%%
%% ============================================================================
%% This work may be distributed and/or modified under the
%% conditions of the LaTeX Project Public License, either version 1.3c
%% of this license or (at your option) any later version.
%% The latest version of this license is in
%% http://www.latex-project.org/lppl.txt
%% and version 1.3c or later is part of all distributions of LaTeX
%% version 2008/05/04 or later.
%%
%% This work has the LPPL maintenance status `maintained'.
%%
%% The Current Maintainers of this work are
%%   Marei Peischl <tuda-ci@peitex.de>
%%   Markus Lazanowski <latex@ce.tu-darmstadt.de>
%%
%% The development respository can be found at
%% https://github.com/tudace/tuda_latex_templates
%% Please use the issue tracker for feedback!
%%
%% If you need a compiled version of this document, have a look at
%% http://mirror.ctan.org/tex-archive/macros/latex/contrib/tuda-ci/doc
%% or at the documentation directory of this package (if installed)
%% <path to your LaTeX distribution>/doc/latex/tuda-ci
%% ============================================================================
%%
% !TeX program = lualatex
%%

\documentclass[
	%ngerman,
	ruledheaders=section,%Ebene bis zu der die Überschriften mit Linien abgetrennt werden, vgl. DEMO-TUDaPub
	class=report,% Basisdokumentenklasse. Wählt die Korrespondierende KOMA-Script Klasse
	thesis={type=master},% Dokumententyp Thesis, für Dissertationen siehe die Demo-Datei DEMO-TUDaPhd
	accentcolor=9c,% Auswahl der Akzentfarbe
	custommargins=true,% Ränder werden mithilfe von typearea automatisch berechnet
	marginpar=false,% Kopfzeile und Fußzeile erstrecken sich nicht über die Randnotizspalte
	%BCOR=5mm,%Bindekorrektur, falls notwendig
	parskip=half-,%Absatzkennzeichnung durch Abstand vgl. KOMA-Sript
	fontsize=11pt,%Basisschriftgröße laut Corporate Design ist mit 9pt häufig zu klein
	logofile=img/tuda_logo, %Falls die Logo Dateien nicht vorliegen
]{tudapub}


% Der folgende Block ist nur bei pdfTeX auf Versionen vor April 2018 notwendig
\usepackage{iftex}
\ifPDFTeX
	\usepackage[utf8]{inputenc}%kompatibilität mit TeX Versionen vor April 2018
\fi

%%%%%%%%%%%%%%%%%%%
%Sprachanpassung & Verbesserte Trennregeln
%%%%%%%%%%%%%%%%%%%
\usepackage[ngerman, main=english]{babel}
\usepackage[autostyle]{csquotes}% Anführungszeichen vereinfacht

% Falls mit pdflatex kompiliert wird, wird microtype automatisch geladen, in diesem Fall muss diese Zeile entfernt werden, und falls weiter Optionen hinzugefügt werden sollen, muss dies über
% \PassOptionsToPackage{Optionen}{microtype}
% vor \documentclass hinzugefügt werden.
\usepackage{microtype}

%%%%%%%%%%%%%%%%%%%
%Literaturverzeichnis
%%%%%%%%%%%%%%%%%%%
\usepackage{biblatex}   % Literaturverzeichnis

%%%%%%%%%%%%%%%%%%%
%Abkürzungsverzeichnis
%%%%%%%%%%%%%%%%%%%
\usepackage[printonlyused, withpage]{acronym}

\usepackage{cleveref}

%%%%%%%%%%%%%%%%%%%
%Paketvorschläge Tabellen
%%%%%%%%%%%%%%%%%%%
%\usepackage{array}     % Basispaket für Tabellenkonfiguration, wird von den folgenden automatisch geladen
\usepackage{tabularx}   % Tabellen, die sich automatisch der Breite anpassen
%\usepackage{longtable} % Mehrseitige Tabellen
%\usepackage{xltabular} % Mehrseitige Tabellen mit anpassarer Breite
\usepackage{booktabs}   % Verbesserte Möglichkeiten für Tabellenlayout über horizontale Linien

%%%%%%%%%%%%%%%%%%%
%Paketvorschläge Mathematik
%%%%%%%%%%%%%%%%%%%
%\usepackage{mathtools} % erweiterte Fassung von amsmath
%\usepackage{amssymb}   % erweiterter Zeichensatz
%\usepackage{siunitx}   % Einheiten

% Diagrams
\usepackage{tikz}
\usetikzlibrary{shapes.geometric, arrows}
\usepackage{epstopdf}

% Code
\usepackage{listings}
\lstset{basicstyle=\ttfamily,breaklines=true}

% Zapf-Dingbats Symbole
\usepackage{pifont}

% TODO Notizen
\usepackage{todonotes}
%\usepackage[disable]{todonotes}
%Formatierungen für Beispiele in diesem Dokument. Im Allgemeinen nicht notwendig!
\let\file\texttt
\let\code\texttt
\let\tbs\textbackslash

% Zapf-Dingbats Symbole
\newcommand*{\FeatureTrue}{\ding{52}}
\newcommand*{\FeatureFalse}{\ding{56}}

\newcommand{\eg}{e.g., }
\newcommand{\ie}{i.e., }

\lstset{
    basicstyle=\ttfamily\small,
    breaklines=true,
    numbers=left,
    stepnumber=1,
    xleftmargin=2.5em,
    frame=lines,
    framexleftmargin=2.5em
}

% ILOC simplifications
\newcommand{\iloc}[1]{\texttt{\small#1}}
\newcommand{\ilocreg}[1]{\iloc{r\textsubscript{#1}}}
\newcommand{\ilocarrow}{$\Rightarrow$}
\newcommand{\iloccmd}[3]{\iloc{#1} & \iloc{#2} & \ilocarrow & \iloc{#3}}
\newcommand{\iloccmdinl}[3]{\iloc{#1}~\iloc{#2}~\ilocarrow~\iloc{#3}}

\newcommand{\aurora}{NEC SX-Aurora TSUBASA-VE}

\newcommand{\tblsection}[1]{\emph{#1}}
\newcommand{\tblitem}[1]{\hspace{0.3cm}#1}

\hyphenation{si-mul-ta-ne-ous-ly}


% Select includes
\includeonly{
    %chapters/about_this_file,
    %chapters/template_usage,
    chapters/background,
    chapters/related_work,
    chapters/evaluation
}

\bibliography{bibliography}

\begin{document}

\Metadata{
	title=Automatic compiler customization for novel hardware,
	author=Janne Wulf
}

\title{Automatic compiler customization for novel hardware}
\subtitle{Automatische Anpassung von Compilern für neuartige Hardwarearchitekturen}
\author[J. Wulf]{Janne Wulf}%optionales Argument ist die Signatur,
\birthplace{Oldenburg in Holstein}%Geburtsort, bei Dissertationen zwingend notwendig
\reviewer{Dr. rer. nat. Stefan Guthe \and  Daniel Thuerck}%Gutachter

%Diese Felder erden untereinander auf der Titelseite platziert.
%\department ist eine notwendige Angabe, siehe auch dem Abschnitt `Abweichung von den Vorgaben für die Titelseite'
\department{inf} % Das Kürzel wird automatisch ersetzt und als Studienfach gewählt, siehe Liste der Kürzel im Dokument.
%\institute{Graphics, Capture and Massively Parallel Computing}
%\group{Arbeitsgruppe}

\submissiondate{August 9, 2021}
\examdate{\today}

%	\tuprints{urn=1234,printid=12345,doi=10.25534/tuprints-1234}
%	\dedication{Für alle, die \TeX{} nutzen.}

\maketitle

\affidavit% oder \affidavit[digital] falls eine rein digitale Abgabe vorgesehen ist.

\tableofcontents

\listoftodos

\chapter{Über diese Datei}
Die Datei \file{DEMO-TUDaThesis.tex} ist ein Template für Abschlussarbeiten im Stil des Corporate Designs der TU Darmstadt.
Sie ist Teil des TUDa-CI-Bundles wurde vom in Teilen tuddesign-Paket von C.~v.~Loewenich und J.~Werner inspiriert.

Sie verwendet die Dokumentenklasse \file{tudapub.cls}, allerdings mit erweiterten Einstellungen. In diesem Dokument werden überwiegend die speziell auf Abschlussarbeiten ausgelegten Möglichkeiten beschrieben. Weitere Konfigurationsmöglichkeiten finden sich in der Datei \file{DEMO-TUDaPub.pdf} \cite{tudapub}.

Es ist voreingestellt, dass eine PDF/A-Datei erzeugt wird. Die beste Kompatibilität hierfür bietet Lua\LaTeX. Bei anderen Compilern kann dies entsprechend der Informationen in DEMO-TUDaPub zu Problemen führen. In diesem Fall sollte entweder der Compiler gewechselt oder \code{pdfa=false} aktiviert werden.

Für weitere Informationen kann ein Blick in die zur Dokumentenklasse gehörigen Dokumentation (tudapub.pdf) hilfreich sein. Sie wird zusammen mit den Quelldateien verteilt.

\minisec{Unterschiede der Demodateien DEMO-TUDaThesis und DEMO-TUDaPhD}
Zwar basieren alle drei DEMO-Dateien auf der Klasse \code{tudapub}, allerdings sind die Basiseinstelungen dem Dokumententyp angepasst.
Für Erläuterungen zu den TUDaPub spezifischen Optionen, sei auf die Datei DEMO-TUDaPub verwiesen.
Da die Basisklasse für beide identisch ist, kann jede Option abgeändert werden. Die Folgende Liste zeigt lediglich die gezeigten Features bei Standardeinstellungen.

\noindent\begin{tabularx}{\linewidth}{@{}p{.25\linewidth}*3{>{\centering\arraybackslash}X}@{}}
	\toprule
	Option&DEMO-TUDaThesis&DEMO-TUDaPhD&DEMO-TUDapub\\
	\midrule
	twoside&\FeatureFalse&\FeatureTrue&\FeatureFalse\\\midrule
	parskip&\FeatureTrue&\FeatureFalse&\FeatureTrue\\\midrule
	Kolophon&\FeatureFalse&\FeatureTrue&\FeatureFalse\\\midrule
	Widmung&\FeatureFalse&\FeatureTrue&\FeatureFalse\\\midrule
	Schriftgröße&11pt&11pt&9pt\\\midrule
	ruledheaders&section&chapter&all\\\midrule
	Basisklasse&scrreprt&scrbook&scrartcl\\\midrule
	thesis&\ttfamily type=bachelor&\ttfamily type=dr, dr=rernat
	&\FeatureFalse\\\midrule
	marginpar&\FeatureFalse&\FeatureFalse&\FeatureTrue\\\midrule
	Affidavit\newline\rlap{(Selbstständigkeitserklärung)}&\FeatureTrue&\FeatureTrue&\FeatureFalse\\\midrule
	abstract&\FeatureFalse&\FeatureTrue&\FeatureTrue\\\midrule
	custommargins&\FeatureTrue&\FeatureTrue&\FeatureFalse\\
	\bottomrule
\end{tabularx}

\chapter{Verwendung}
Die Klasse kann wie für Dokumentenklassen üblich eingebunden werden
\begin{verbatim}
\documentclass[thesis]{tudapub}
\end{verbatim}
Die Option \code{thesis} wechselt hierbei in den Modus, der spezielle Features für Abschlussarbeiten freischaltet, die in diesem Dokument beschrieben werden.

Darüber hinaus lässt sich die Klasse verwenden wie die Standard-KOMA-Script-Klasse, auf der sie basiert.
Voreingestellt ist hierbei \code{scrreprt}.

Allgemein bietet \KOMAScript{} viele Möglichkeiten zu Anpassungen. Wie in der tudapub-Demo-Datei beschrieben, können hier jedoch nicht alle erläutert werden, ein Blick in die offizielle Dokumentation ist daher häufig hilfreich \cite{scrguide}.

\section{Sprachanpassung}
Der Modus für Abschlussarbeiten setzt einige sprachabhängige Bezeichnungen.
Teilweise ist Deutsch für diese Elemente als Hauptsprache vorgeschrieben (z.\,B. die Selbstständigkeitserklärung). Für die korrekte Verarbeitung wird daher ein Paket zur Sprachanpassung benötigt.
TUDa-CI verwendet hierfür das babel-Paket.

Dies wird jedoch nicht automatisch geladen, da hierfür die Konfiguration der Sprachen bekannt sein müsste. Die Demo-Dateien für Abschlussarbeiten (\file{DEMO-TUDaThesis.tex}/""\file{DEMO-TUDaPhD.tex}) laden hierfür die Konfiguration:
\begin{verbatim}
	\usepackage[english, main=ngerman]{babel}
\end{verbatim}
Diese ist für ein Dokument mit Deutsch als Hauptsprache und Englischen Elementen.
Die Hauptsprache wird als Wert der Option \verb+main=+ übergeben.
Das Laden von \verb+ngerman+ wird in den Fällen, in denen es von TUDa-CI benötigt wird, automatisch ausgelöst.
Für eine bessere Übersichtlichkeit ist es dennoch hilfreich es dort aufzuführen.

Falls die Hauptsprache nicht Deutsch ist, wäre daher die folgende Konfiguration sinnvoll:
\begin{verbatim}
	\usepackage[ngerman, main=<Hauptsprache>]{babel}
\end{verbatim}

\section{Übergabe der Titelinformationen}

Die Titelinformationen werden analog zur klassichen Titelerzeugung mit \verb+\maketitle+ übergeben. Allerdings wurden die Felder um ein paar speziellere Daten erweitert. Sofern nicht anders angegeben, verfügen alle Makros über ein notwendiges Argument für die Datenübergabe, z.\,B.
\begin{verbatim}
\title{\LaTeX{} im Corporate Design der TU Darmstadt}
\end{verbatim}
Es ist zu beachten, dass für die Erzeugung der Titelseite nach Übergabe aller Daten \verb+\maketitle+ aufgerufen werden muss.

\begin{description}\setkomafont{descriptionlabel}{\ttfamily\textbackslash}
	\item[title] Titel, wird in sehr großer Schrift im obersten Block der Titelseite platziert. Die Schriftgröße ist aufgrund der Häufigkeit für lange Titel kleiner gewählt als für andere Publikationen.
	\item[subtitle] Untertitel. Dieses Feld kann alternativ für eine Übersetzung genutzt werden.
	\item[author] Der Autor/dir Autoren. Mehere Autoren werden durch \verb+\and+ getrennt.
	\item[studentID] Matrikelnummer. Nach den Vorgaben des Templates ist diese Angabe immer optional.
	\item[birthplace] Geburtsort. Angabe ist bei Dissertationen notwendig.
	\item[reviewer] Gutachter. Mehrere Gutachter werden, wie Autoren durch \verb+\and+ getrennt. Die Nummerierung läuft von links nach rechts.
	\minisec{Änderung des Bezeichners}
	Die Änderung des Bezeichners ist über ein optionales Argument möglich:
	\begin{verbatim}
		\reviewer[Ersatzbezeichner]{Name1 \and Name2}
	\end{verbatim}
	Um die numerische Benennung abzuändern ist es zusätzlich möglich statt dem Ersatzbezeichner eine Kommaliste zu übergeben:
	\begin{verbatim}
		\reviewer*[Bezeichner1, Bezeichner2]{Name1 \and Name2}
	\end{verbatim}
	In diesem Fall entfällt die Nummerierung vor dem Bezeichner. Soll z.\,B. den Formulierungen der Promotionsordnung entsprochen werden, gilt:
	\begin{verbatim}
		\reviewer[Erstreferent\_in,Koreferent\_in]{Name1 \and Name2}
	\end{verbatim}
	Für die Erstellung Fachbereichsspezifischer Templates existiert hierfür auch ein Makro, dass ohne den Aufruf von \verb+\reviewer+ Änderungen zulässt.
	\begin{verbatim}
	\setupReviewName{Ersatzwort für „Gutachten“}
	\end{verbatim}
	Setzt die ersten beiden Bezeichner. Alternativ ist es auch möglich Positionen einzeln zu benennen \verb+\setupReviewName[1]{Erstferent}+, eine Übergabe als Komma-Liste ist als \verb+\setupReviewName*{Bezeicher1,Bezeicher2}+ möglich.
	\item[institution] Einrichtung. Dieser Eintrag, wie auch die beiden folgenden, werden unterhalb des Logos auf der Titelseite platziert.
	\item[department] Fach-/Studienbereich, allerdings ist die oben genannte Option zu bevorzugen. Die Verarbeitung des Arguments erfolgt jedoch analog.

	Dieses Makro verfügt jedoch zusätzlich über die Möglichkeit abweichende Einträge gegenüber den Vorgaben anzugeben. Insbesondere wenn eine gesonderte Formulierung gegenüber der voreingestellten \enquote{im Fachbereich} und ihren Varianten notwendig ist. Hierfür liefert \code{\textbackslash{}department} ein optionales Argument:

	\begin{verbatim}
	\department[Ersatztext]{Kürzel/Bezeichnung}
	\end{verbatim}
	Zusätzlich gibt es ab Version 2.01 auch die Möglichkeit den gesamten Text \enquote{im Fachbereich <Bereichsbezeichnung>}, sowie die Angabe in der Infobox auf der Titelseite zu ersetzen. Dies geschieht über die gesternte Variante:
	\begin{verbatim}
	\department*[Text für die Box]{Text zwischen Typ und Autor}
	\end{verbatim}
	\item[group] Arbeitsgruppe.
	\item[submissiondate] Datum der Einreichung
	\item[examdate] Datum der Disputation
	\item[date] Beliebiges Datum. Wird über \verb|datename| bezeichnet.
	\item[publishers] Wird hier für die Ortsangabe verwendet und ist mit \enquote{Darmstadt}, bzw. \enquote{Darmstadt -- D17} (bei Dissertationen) vorbelegt.
	\item[tuprints] \label{page:tuprints}Übergabe der Daten, sofern das dokument über tuprints Veröffentlicht werden soll.
	\begin{verbatim}
	\tuprints{urn=1234, printid=12345, doi=10.25534/tuprints-1234}
	\end{verbatim}
	Falls das Argument kein Gleichheitszeichen erkennt, wird der Wert als \code{printid} gesetzt und keine URN angegeben.

	\item[titleimage] Hier kann Code übergeben werden, der den farbigen Block im unteren Teil der Titelseite ersetzt. Als Maße können hier die Längen \verb+\layerwidth+ und \verb+\layerheight+ verwendet werden. Sie passen sich dem Verfügbaren Platz an. Für ein Beispiel sei auf die TUDapub-Dokumentation verwiesen.
	\item[titleintro] Ab Version 2.03 kann zusätzlich über diesen Hook ein beliebiger Text direkt nach dem Untertitel und vor den automatischen Informationen ergänzt werden.
	\item[titleaddendum] Wie \code{\tbs{}titleintro} jedoch als letztes Element des Blocks.
\end{description}

\section{Weitere Macros}
Das Makro \verb+\affidavit+ erzeugt eine Selbstständigkeitserklärung mit Unterschriftenzeile. Hier wird der oben übergebene Name/Signatur eingefügt.
In diesem Dokument findet sich das Affidavit direkt nach der Titelei.

Ab Version 3.06 verfügt \verb+\affidavit+ zusätzlich über ein optionales Argument, über das der Modus eingestellt werden kann.
Es besteht die Unterscheidung zwischen \verb+digital+ und \verb+print+. Hier wird entsprechend der Informationen unter \url{https://www.tu-darmstadt.de/studieren/studierende_tu/studienorganisation_und_tucan/hilfe_und_faq/artikel_details_de_en_37824.de.jsp} der entsprechende Text automatisch ausgewählt.

\begin{verbatim}
\affidavit[digital]
\end{verbatim}

Da für Dissertationen bisher keine Option der rein digitalen Abgabe besteht, entfällt dort diese Unterscheidungsmöglichkeit.

Es besteht zusätzlich die Möglichkeit ein anderssprachiges Affidavit als Ergänzung mit abzudrucken. Um die Struktur und die ggf. notwendige Sprachumschaltung zu erledigen, existiert hierfür ab Version 2.03 eine Umgebung:

\begin{verbatim}
\begin{affidavit*}[Babel-Sprachoption]{Überschrift}
Text
\end{affidavit*}
\end{verbatim}

Diese Variante verfügt bewusst über keine Unterschriftenzeile, da diese Version laut Verständnis der Entwickler keine rechtliche Verbindlichkeit besitzt.

Die Umgebung kann jedoch auch für besondere Formen der Erklärung genutzt werden. In diesem Fall kann eine zusätzliche Signaturzeile über
\begin{verbatim}
\AffidavitSignature[Stadt]
\end{verbatim}
hinzugefügt werden. Die Vorbelegung für Stadt ist hierbei \enquote{Darmstadt}.

\section{Layout-Optionen mit Verstoß gegen das Corporate Design}
Die Zeilenlängen sind laut Corporate Design aus typografischer Sicht zu lang.
Daher existiert die Klassenoption \code{custommargins}, die für dieses Dokument aktiviert wurde.

Bei Verwendung einer Bindekorrektur wird diese nicht automatisch auch auf der Titelseite eingefügt. Für diesen Fall wurde mit Version 3.0 zusätzlich die Option \code{BCORtitlepage} hinzugefügt. Falls diese aktiviert wird, nimmt die Titelseite den Wert der Typearea Option \code{BCOR} auf der ersten Seite als Zusatz zum linken Rand hinzu.

Die Option \code{custommargins} verfügt ab Version 1.10 auch über den Wert \code{geometry}. Damit können die Ränder auch durch einen Aufruf von \code{\tbs{}geometry} vor Beginn des Dokuments manuell angepasst werden.

Hierbei ist zu beachten, dass die Einstellungen als Ausgangspunkt den Voreingestellten Satzspiegel nutzen (je nach Option mit Randnotizspalte oder ohne). Es ist möglich diese Optionen vor den eigenen mit zurückzusetzen:
\begin{verbatim}
\geometry{
	reset,
	<Eigene Anpassungen>
}
\end{verbatim}
Die gilt insbesondere für die Optionen \code{includehead}, \code{includefoot} und \code{includemp}.

Diese Variante wird auf Wunsch zur Verfügung gestellt, allerdings wird darauf hingewiesen, dass manuelle Randeinstellungen oft nicht zu einem harmonischen Satzspiegel führen.

Auch ist das Standard-Layout der Kolumnentitel wenig vorteilhaft, da die Kolumnentitel damit local größer sein können als die eigentliche Überschrift.


Dadurch werden die Ränder nicht fest definiert, sondern auf Basis des typearea-Paketes optimiert.

Wenn die option \code{marginpar=true} gesetzt bleibt, ragen die Kopf- und Fußzeile über die Marginalspalte hinaus. Aus ästhetischen Gründen wird daher empfohlen in diesem Fall die Kopf- und Fußzeile  mit \code{marginpar=false}  auf den Textbereich zu beschränken.


Darüber hinaus kann über
\begin{verbatim}
\pagestyle{TUDa.headings}
\end{verbatim}
ein einfacherer Seitenstil ausgewählt werden, der die Nutzung mit lebenden Kolumnentitel erheblich vereinfacht.


\section{Spezielle Optionen für Abschlussarbeiten}
Die Klasse unterstützt alle Optionen der \file{tudapub}-Klasse. Darüber hinaus besteht über Wertzuweisung der Option \code{thesis} die Möglichkeit spezielle Einstellungen zu wählen.
Es ist prinzipiell möglich die Optionen auch direkt als Optionen zur \file{tudapub}-Klasse zu übergeben, allerdings ist dies aufgrund der schlechteren Übersicht nicht zu empfehlen.

Für dieses Dokument wurden beispielsweise die Optionen als
\begin{verbatim}
thesis={type=drfinal,dr=phil}
\end{verbatim}
übergeben.

Im folgenden findet sich die Bedeutung der einzelnen Optionen:
\begin{description}
\item[type=<Wert>] Auswahl des Typus. Dieser wird auf die Titelseite gesetzt und wählt zudem aus welche Informationen für die Titelseite zwingend übergeben werden müssen.
	Es stehen die folgenden Werte zur Verfügung (die Werte in Klammern sind die notwendigen Titeldaten):
	\begin{itemize}
	\item \code{sta}: Studienarbeit (title, author, date)
	\item \code{diplom}: Diplomarbeit (title, author, submissiondate, reviewer, department)
	\item \code{bachelor}: Bachelorarbeit (title, author, submissiondate, department, reviewer)
	\item \code{master}: Masterarbeit (title, author, submissiondate, department, reviewer)
	\item \code{pp}: Project-Proposal  (title, author, date, department)
	\item \code{dr}: vorgelegte Dissertation (title, author, submissiondate , birthplace, department, reviewer)
	\item \code{drfinal}: genehmigte Dissertation (title, author, submissiondate,examdate, birthplace, department, reviewer)
	\end{itemize}
	Wird ein Typus angegeben, der nicht erkannt wird, so wird der Text direkt übergeben. Notwendige Titelfelder über den Titel hinaus gibt es in diesem Fall nicht.
\item[dr=<Kürzel>] Lädt einen der vordefinierten Texte für die Titelseite. Als Werte stehen bislang \code{rernat}, \code{ing} und \code{phil} zur Verfügung. Zum Beispiel lädt der Wert \code{phil}:
	\begin{quote}
	Zur Erlangung des Grades eines Doktor der Philosophie (Dr.\,phil.)
	\end{quote}
	Sofern keiner dieser Werte dem angestrebten Titel entspricht, kann ein Text direkt übergeben werden.
	\begin{verbatim}
	\drtext{Zur Erlangung des Grades \ldots}
	\end{verbatim}
\item[department=<Kürzel>] Die Fachbereiche sind fest als Textbausteine in Deutscher sowie Englischer Sprache hinterlegt. Diese Option ermöglicht die Auswahl als Dokumentenklassenoption. Aus Kompatibilitätsgründen kann jedoch auch das Makro \code{department}-Makro hierfür genutzt werden. Zur Verfügung stehen:\par
	\begin{tabular}{@{}l@{${}\to{}$}l@{}}
		arch  & Architektur\\
		bauing& Bau- und Umweltingenieurwissenschaften\\
		bio   &Biologie\\
		chem  &Chemie\\
		etit  &Elektrotechnik und Informationstechnik\\
		gugw  &Gesellschafts- und Geschichtswissenschaften\\
		humanw&Humanwissenschaften\\
		inf   &Informatik\\
		mb    &Maschinenbau\\
		matgeo&Material- und Geowissenschaften\\
		math  &Mathematik\\
		phys  &Physik\\
		wi    &Rechts- und Wirtschaftswissenschaften
	\end{tabular}

	Neben den Fachbereichen existieren für Abschlussarbeiten, die keine Dissertationen sind auch Studienbereiche.
	Falls das Kürzel nicht als Fachbereich hinterlegt ist, wird automatisch auf die Studienbereiche geprüft. Die Studienbereiche haben die folgenden Kürzel:

	\begin{tabular}{@{}l@{${}\to{}$}l@{}}
		ce&Computational Engineering\\
		ese&Energy Science and Engineering\\
		ist&Information Systems Engineering\\
		mech&Mechanik\\
		metro&Mechatronik
	\end{tabular}

	Falls etwas anderes als eines dieser Kürzel übergeben wird, wird der Text direkt verwendet und eine entsprechende Warnung ausgegeben.

	Die Auswahl der Fachrichtung erzeugt zusätzlich eine Box auf der Titelseite unterhalb des Logos. Falls diese automatische Erstellung nicht gewünscht ist, kann dies über die Option \code{instbox=false} deaktiviert werden.
\item[ignore-missing-data] Diese Option ist ein Schalter, der es ermöglicht die Fehlermeldung über nicht übergebene Titeldaten auszuschalten. In diesem Fall wird lediglich eine Warnung erzeugt, falls die angegeben Daten nicht mit den Anforderungen übereinstimmen.
\end{description}

\minisec{Abweichung von den Vorgaben für die Titelseite}
Da es möglich sein kann von dieser Vorgabe abzuweichen, existiert für Sonderfälle die Dokumentenklassenoption \code{instbox=false}. Damit wird die automatische Verarbeitung der Daten für die Boxen auf der der Titelseite unterdrückt. In diesem Fall ist der Autor jedoch selbst für die Einhaltung der Vorschriften verantwortlich. Weitere Informationen zur Konstruktion der Boxen findet sich in den Verwendungshinweisen zu Basisklasse TUDaPub. Zusätzlich sei auf die Möglichkeiten des \code{\textbackslash{}department}-Makros verwiesen, sofern die Abweichung sich auf den Text beschränkt.

\section{Erhöhter Zeilenabstand -- Informationen zum setspace-Paket}
Sofern die Vorgaben es erfordern, ist es möglich mit dem setspace-Paket den Durchschuss zu erhöhen. Allerdings beeinflusst dies natürlich sämtliche Zeilenabstände. Ein erhöhter Zeilenabstand sollte daher erst nach der Titelseite aktiviert werden. Allgemein ist es jedoch empfehlenswert auch für Verzeichnisse und sonstige Sonderelemente außerhalb des Fließtextes auf bei normalen Einstellungen zu bleiben.

Setspace liefert hierfür die Möglichkeit, das Paket ohne Optionen zu laden und später über Makros, wie \code{\tbs{}onehalfspacing} das Umschalten zu verzögern. Alternativ kann auch durch die Umgebungen, wie \code{singlespace} lokal wieder zum Normalzustand gewechselt werden, sofern dies erforderlich ist.
\chapter{Background}
% NOTE: Assume that the reader has common computer science knowledge!
In this chapter we are introducing knowledge which is required to understand the content of this thesis.
In \cref{sec:bg:compilers} we give an overview about how a compiler is typically implemented and which problems it has to solve that we want to tackle.
The \crefrange{sec:bg:compilers:frontend}{sec:bg:compilers:optimizer} give a superficial overview of the first two phases of a typical compiler.
\Cref{sec:bg:compilers:backend} gives a more detailed introduction to compiler back ends, which we aim to optimize in this thesis.
\todo{explain also the other sections}

%\section{Motivation?}
%\cite{goodman1988code} state the important interdependence between instruction scheduling and register allocation.

\section{CPU Functionality}
ISA\\
Memory and Registers\\
How do CPUs work in general? -> Fetch, Decode, etc.\\
In-order vs out-of-order architectures

\section{Compilers}
\label{sec:bg:compilers}
%What is a compiler? \\
% \todo{This is probably too basic and not required. Might maybe be used in introduction, though}
% In the early days of computer programming, computers were programmed in assembly languages.
% These languages are exclusive to specific processor architectures and only provide the instructions that are available by the \ac{isa} of the processor.
% This approach has several disadvantages.
% Lots of complicated code, that is hard to understand, is written to express even simple programs.
% Also, each program has to be rewritten to execute on a processor with another architecture.
%
% Nowadays computers are programmed---almost exclusively---in high-level programming languages like C/C++, Java, Python, etc.
% The processor is not able to execute code written in such high-level programming languages, though.
% For this reason a compiler has to translate the code into a format that the processor can understand.
Making computer programs, that are written in high-level programming languages~(\eg C/C++, Java, Rust), executable on a specific machine is not a trivial task.
Compilers are only one piece in the tool-chain required to make a program executable.
The compiler translates the high-level language into assembly language, which is translated into object code by the assembler.
Basic functionality like allocating memory or outputting strings on the screen is implemented in a standard library.
The object code of the standard library and potentially other libraries are linked together with the translated program by the linker.
There is more required to execute the code on a specific machine (\eg a runtime library), but explaining this would go beyond the scope of this thesis.

%Why are compilers important? \\
%What are the different tasks a compiler has? \\
The pure translation of the program is only one of the tasks a compiler has to fulfill.
It also has to assure that the program is written in correct syntax of the high-level language.
Most compilers will also optimize the given code and the translated code since a simple one-to-one translation would have a very poor performance.
Eventually there has to be a mapping from variables and to the main memory and the processors registers.
These are all by itself complex problems which are handled by a compiler.

%How are compilers usually implemented? (Front-End, ...)\\
Compilers are usually implemented in different phases to seperate the different tasks and have a structured approach.
A common approach to structure a compiler is by having a front end~(\cref{sec:bg:compilers:frontend}), an optional optimization~(\cref{sec:bg:compilers:optimizer}) and a back end phase~(\cref{sec:bg:compilers:backend}).
These phases are explained in more detail in the following sub sections.

\subsection{Front End}
\label{sec:bg:compilers:frontend}
The front end phase is the first step in the translation process.
Its implementation is dependent on the source language that has to be translated.
A typical front end includes a scanner, a syntax checker/parser, a context-sensitive analysis and translation into a \ac{ir}.
The scanner translates a stream of characters into a stream of tokens that are classified as parts of the source language.
These tokens are then taken by the parser and are checked against the grammer defined by the source language.
Even with a syntactically correct program, there can still be errors in the code, \eg assignments of incompatible types.
These are checked during the context-sensitive analysis phase.
Eventually, the source code is translated into some kind of a \ac{ir} which will be used as input to the optimizer and the back end.

There might be additional steps required depending on the source language.
C/C++ compilers, for example, use a preprocessor to replace macros like \lstinline[language=C]|#include| and \lstinline[language=C]|#define| with their actual values.

\subsection{Optimizer}
\label{sec:bg:compilers:optimizer}
The usage of a \ac{ir}, not only abstracts away the source language and the target hardware, but also permits to apply more passes in the compilation process.
These addition passes transform \ac{ir} to \ac{ir}.
Note that this step is optional and not required to produce correct translations.
The purpose of this step is to optimize the code, in a source and target independent manner, for more efficient execution.
Efficient can mean different things here. % \eg faster, lower memory usage, lower energy consumption.
The transformed code might produce, \eg a faster program, a program that is smaller in size, or a program with less power consumption.
There exist a large amount of optimization passes in most compilers.
They range from rather simple optimizations, like replacing constant variables with their actual value, to more advanced ones that might, for example, simpliy computations with the rules of algebra.

\subsection{Back End}
\label{sec:bg:compilers:backend}

\subsubsection{Instruction Scheduling}
Add Example: e.g. see \url{https://youtu.be/brpomKUynEA?t=271}

\subsubsection{Register Allocation}

\section{LLVM Compiler Infrastructure}
\subsection{Intermediate Representation}
\subsection{Instruction Selection DAG}
\missingfigure[figwidth=\linewidth]{Selection DAG}
\subsection{Pre-RA-Scheduling}
Welche gibt es?\\
Wie funktionieren sie?\\
Welche Infos nutzen sie?\\
\subsection{Post-RA-Scheduling}

\section{Reinforcement Learning}

\chapter{Related Work}
\label{sec:rw}
In this chapter, we survey the existing relevant research in the topic area of this thesis. 
We present the fundamental research and newer data-driven developments in the related fields of instruction scheduling (\Cref{sec:rw:instruction-scheduling}) and register allocation (\Cref{sec:rw:register-allocation}).
Further, we discuss some other relevant works in the fields of machine learning based compiler optimizations, runtime estimation, code feature extraction and machine learning approaches on other scheduling tasks (\Cref{sec:rw:other}).

\section{Instruction Scheduling}
\label{sec:rw:instruction-scheduling}
\subsection{Classical Approaches}
Scheduling problems appear in many fields.
This is why general scheduling is a topic with much existing research.
Also the research on instruction scheduling has a long history.

Algorithms exist that can generate perfect instruction schedules for simple situations with perfect information.
The requirements are met for architectures with only one functional unit and uniform instruction latencies.
The best-known algorithms in this field is the Sethi-Ullman labelling algorithm~\cite{sethi1970generation} and the work by \citeauthor{proebsting1991linear}~\cite{proebsting1991linear}.

However, these conditions are not present in modern processors.
In more complex situations, the instruction scheduling problem is NP-complex~\cite{hennessy1983postpass}.
Modern processors use pipelines to achieve instruction parallelism, see \Cref{sec:bg:cpu}.
Consequently, most instruction schedulers that are used nowadays are based on the list scheduling framework, which was proposed by \citeauthor{landskov1980local}~\cite{landskov1980local}.
The algorithms, that follow this approach, are better able to generate instruction schedules for pipelined processors.
\citeauthor{heller1961sequencing}~\cite{heller1961sequencing} published an early work on how to approach instruction scheduling for these processors.
Much research on further developments of the list scheduling was published~\cite{bernstein1991global,gibbons1986efficient,hennessy1983postpass}.

As elaborated in \Cref{sec:bg:cpu}, the available information on instruction latencies is mostly uncertain.
The reasons are instruction-level parallelism and uncertain memory latencies.
One way to approach this problem, is balanced scheduling~\cite{kerns1993balanced,lo1995improving}.
Another proposed idea, was to use stochastic instruction scheduling~\cite{schielke2000stochastic}.

Instruction scheduling typically works on a basic block level.
This also means that transitions between basic blocks are not scheduled to work well together.
However, research exists on extending the scope to greater regions~\cite{fisher1981trace,bernstein1991global,hwu1993superblock}.

\subsection{Machine Learning Approaches}
The first work that combined data-driven methods with instruction scheduling was a patent by \citeauthor{tarsy1994method}~\cite{tarsy1994method}, filed in \citeyear{tarsy1994method}.
They optimize weights that are used in cost-based heuristics.
These heuristics are used in list scheduling for pipelined processors.

List schedulers usually have multiple heuristics that are used for choosing an instruction from the list of available instructions.
The selection is based on a weighted sum of the heuristics.
\citeauthor{beaty1996using}~\cite{beaty1996using} published a work in \citeyear{beaty1996using}, in which they have used a genetic algorithm to learn weights for different heuristics.
They achieved a 5\% performance increase compared to a random scheduler on three architectures.

\citeauthor{moss1997learning}~\cite{moss1997learning} trained a function that would prefer one instruction over another when presented the previosuly scheduled instructions.
They used decision trees, look-up tables, ELF function approximations, and feed-forward neural networks.
The decision tree performed best and found often the optimal schedule.
However, they only used simulations and limited the basic block length to 10 instructions.

A reinforcement learning and a search heuristic were proposed by \citeauthor{mcgovern1999scheduling}~\cite{mcgovern1999scheduling,mcgovern2002building}.
Their reinforcement learning heuristic sometimes found a better instruction schedule than their baseline.
The long-running search approach found a better instruction schedule every time.
However, their baseline was only a random instruction scheduler and they have only used simulation results.

\citeauthor{russell2006learning}~\cite{russell2006learning} used decision trees to create heuristics to improve instruction scheduling decisions.
They showed, that they generated better instruction schedules 7.8 times more often than the compared heuristics.
Their results are also based on simulations only.

A newer work in this field was published by \citeauthor{jain2019learning}~\cite{jain2019learning}.
They trained a neural network to imitate the instruction schedules by the GCC compiler.
However, this approach is limited by the performance of the GCC instruction scheduler and cannot exceed it.

We conclude, that the machine learning based approaches mostly performed well in theory, but only against weak random baselines.
The only work, that we found that was evaluated on hardware was~\cite{beaty1996using}.

% \citeauthor{cavazos2004inducing} showed that instruction scheduling only makes a difference on some basic blocks.
% They used decision trees for selecting basic blocks for instruction scheduling~\cite{cavazos2004inducing}.

\section{Register Allocation}
\label{sec:rw:register-allocation}
We have disussed the implications of the instruction scheduling phase on the register allocation \Cref{sec:bg:compilers:backend}.
This interdependence was also shown by \citeauthor{goodman1988code}~\cite{goodman1988code}.
\citeauthor{lavrov1962store} showed the connection between the graph-coloring problem and register allocation and thus, the NP-completeness~\cite{lavrov1962store}.
The first graph-coloring based algorithm implemented in a compiler by \citeauthor{chaitin1982register}~\cite{chaitin1982register}.

In the field of register allocation also appeared research that builds the connection to data-driven methods.
\citeauthor{das2019deep} used a deep learning approach to solve the graph coloring problem~\cite{das2019deep}.
The newer and naturally better fitting approach with graph neural networks was used by \citeauthor{lemos2019graph}~\cite{lemos2019graph} to solve the graph coloring problem. 

\section{Other}
\label{sec:rw:other}
\subsection{Compiler Optimizations with Machine Learning}
% \cite{mammadli2020static,haj2020neurovectorizer,huang2019autophase,qiao2019loop}
Machine learning approaches are also applied to optimize other parts of the compilation process.
Deep reinforcement learning was successfully applied to the phase-ordering problem by \citeauthor{mammadli2020static}~\cite{mammadli2020static} and \citeauthor{huang2019autophase}~\cite{huang2019autophase}.
Phase-ordering means to select the compilers optimization passes and define its execution order (see \Cref{sec:bg:compilers:optimizer} for information on the optimization phase).
Deep reinforcement learning was also used by \citeauthor{haj2020neurovectorizer}~\cite{haj2020neurovectorizer} to translate loops into vector processing instructions (SIMD).
\citeauthor{wang2009mapping}~\cite{wang2009mapping} used machine learning to predict the optimal number of threads and the optimal scheduling policy for OpenMP parallelized loops.
For more works, see the surveys~\cite{wang2018machine,ashouri2018survey}.

\subsection{Runtime Estimation}
Various tools for throughput and runtime estimation exist, like Ithemal~\cite{mendis2019ithemal}, llvm-mca\footnote{https://llvm.org/docs/CommandGuide/llvm-mca.html}, and Intel Architecture Code Analyzer (IACA)\footnote{https://software.intel.com/content/www/us/en/develop/articles/intel-architecture-code-analyzer.html}.
However, the listed tools only work with the x86 architecture, which we do not use.
Especially the Ithemal~\cite{mendis2019ithemal} project is interesting as they use a neural network to predict the runtime from the basic block.
That means they learned to extract features from the basic block to predict its runtime.
% This could also be interesting in the instruction scheduling task.

% Code representation learning
\subsection{Feature Extraction from Code}
% \cite{ben2018neural,cummins2021programl,brauckmann2020compiler}
The previously cited works on data-driven machine learning optimizations have or might benefit from research whose goal it is to automatically extract features from code.
A similar approach to the word2vec~\cite{mikolov2013efficient} approach in the \ac{nlp} area was proposed by~\cite{ben2018neural,alon2019code2vec}.
\citeauthor{cummins2021programl} developed a method to extract features from code, that is based on graph structures~\cite{cummins2021programl}.
The work proposed by \citeauthor{brauckmann2020compiler}~\cite{brauckmann2020compiler} works similarly, they also work on graph structures and use graph neural networks to extract features.

\subsection{Other Scheduling Tasks with Machine Learning}
\citeauthor{mao2019learning}~\cite{mao2019learning} have used a deep reinforcement learning approach to schedule data-processing jobs onto compute clusters.
This work is interesting because the jobs have dependencies on each other which are represented in a \ac{dag}, just like the instructions in the instruction scheduling problem.

% \subsection*{Register Allocation for Intel Processor Graphics}\cite{chen2018register}

% \section{Compiler Optimization Phase Ordering}
% %Static neural compiler optimization via deep reinforcement learning~\cite{mammadli2020static}\\
% % - Uses only static information extracted from IR 
% % - IR embedded by using ben2018neural
% % - Use deep q learning
% % - training is executed by running the modified IR and measure the speedup 
% % - reward is defined as ln(T(s_t)/T(s_t+1)) with T being the runtime


% \subsection*{Autophase: Compiler phase-ordering for hls with deep reinforcement learning}\cite{huang2019autophase}

% \section{Code Representation}
% \label{sec:rw:code-representation}
% For making use of data driven techniques in the area of compiler optimization, it is required to somehow extract features from the code to make it accessible for data driven algorithms.
% Older works usually made use of approaches that used hand-tuned features.
% \todo{Maybe add references used in https://chriscummins.cc/u/ed/phd-thesis.pdf (3.3.2.1)}

% Recent works are inspired by the advances in the the field of \ac{nlp}, which are caused by neural networks and continuous distributed vectors (referred to as embeddings) \eg, word2vec~\cite{mikolov2013efficient}. 
% Although, human language is different from codes of programming languages in many aspects, embeddings prove to be useful in code related tasks, too.

% Code inputs may be used directly in a high-level programming language or in an \ac{ir} (\eg, LLVM-IR~\cite{LLVM:CGO04}).
% The advantage of using an \ac{ir} is that it is independent of the source programming language and the target architecture.

%Overviews:
%\begin{itemize}
%    \item ProGraML Paper under Motivation
%    \item https://chriscummins.cc/u/ed/phd-thesis.pdf (3.3.2.1)
%    \item https://arxiv.org/pdf/1904.03061.pdf
%\end{itemize}

% Most approaches for representing high-level language code use some sort of the \ac{ast} in combination with various learning mechanisms.
% %code2vec: Learning distributed representations of code \cite{alon2019code2vec}
% % - works on source code
% % - sensitive to identifier names
% \citeauthor{alon2019code2vec}~\cite{alon2019code2vec} used paths of the \ac{ast} in combination with a Attention Neural Network model.
% Others have used the \ac{ast} in combination with Gated Graph Neural Networks~\cite{ye2020deep, allamanis2017learning}, with Support Vector Machines~\cite{park2012using} or with \ac{lstm} Networks for tree structures~\cite{dam2018deep}.
%\cite{ye2020deep} % Deep Program Structure Modeling Through Multi-RelationalGraph-based Learning, graph-based deep learning (Gated Graph Neural Networks), AST
%\cite{allamanis2017learning} % LEARNING  TO REPRESENT PROGRAMS WITH GRAPHS, graph-based deep learning (Gated Graph Neural Networks), AST
%\cite{dam2018deep} % A deep tree-based model for software defect prediction, AST, tree-LSTM
%\cite{park2012using} % Using Graph-Based ProgramCharacterization for Predictive Modeling, SVM

% \subsubsection{Neural code comprehension: A learnable representation of code semantics~\cite{ben2018neural}}
% - Defines embedding space inst2vec
% - Encodes LLVM-IR, independent of source programming language
% - Leveraging data- and control flow (Contextual Flow Graphs)
% - Use RNN
% - Analyze the embeddings qualitatively using analogies and clustering, and evaluate the learned representation on three different high-level tasks
% - See description in ProGraML Paper under Motivation
% With Neural Code Comprehension (inst2vec)~\cite{ben2018neural}, Ben-Nun et al. defined an embedding space for the LLVM-IR.
% Relevant information to discover code semantics are data and control flow. 
% To emphasize the semantics, the data and control flow are represented in a novel graph structure, called \acp{xfg}.
% %Before building the \ac{xfg}, the LLVM-IR code is split into basic blocks, so diverging control flow is eliminated.
% The context of an individual statement, with size $N$, is defined as the statement and its graph neighbors that are connected by a path of length $N$.
% This statement is then mapped to its embedding by using the skip-gram model~\cite{mikolov2013distributed}, which are known to work good in \ac{nlp} tasks.
% The \ac{xfg} captures features like data and control dependence's, instructions and data types, which are important for our task.

% \subsection*{ProGraML: Graph-based Deep Learning for Program Optimization and Analysis}\cite{cummins2021programl}
%\begin{itemize}
%    \item approach is insensitive to identifier names and preserves operand order and type information
%    \item compared to inst2vec, it can do the same plus preserve operand order, important to distinguish non-commutative ops
%    \item represent programs as directed multigraphs where statements, identifiers, and immediate values are vertices, and relations between vertices are edge
%    \item encode IR into a graph which will be consumed by a Message Passing Neural Network to execute some task
%\end{itemize}

% \subsection*{Compiler-based graph representations for deep learning models of code}\cite{brauckmann2020compiler}
% \todo{Find Paper PDF and write text}

%IR2Vec: A Flow Analysis based Scalable Infrastructure for Program Encodings \cite{keerthy2019ir2vec}
%\begin{itemize}
%    \item Abstracts away the width of the datatype
%\end{itemize}
% IR2Vec~\cite{keerthy2019ir2vec} is another approach that maps an \ac{ir} to a embedding space.
% However, the datatype size, which is important for code optimizations, is abstracted away during the embedding process.

% \section{Applied Machine Learning on Code}
% \label{sec:rw:applied-code-ml}
% Machine learning approaches are used in newer research for code analysis and for compiler optimizations.
% Before machine learning algorithms can be applied, the code must be transformed into some format that the algorithm can consume.

% Various approaches exist that are inspired by the advances in the \ac{nlp} field.
% Much success in the \ac{nlp} field is based on word embeddings~\cite{mikolov2013efficient}.
% Embeddings map words into a high-dimensional continous vector space, such that words with a similar meaning are close to each other.
% \citeauthor{alon2019code2vec}~\cite{alon2019code2vec} used paths of the \ac{ast} in combination with a Attention Neural Network model to map Java code snippets into an embedding space.
% A more broadly applicable approach is to map from instructions of the LLVM \ac{ir} into an embedding space, because it is independent of the source programming language and the target hardware architecture.
% A recent work, that implements this mapping is from \citeauthor{venkatakeerthy2020ir2vec}~\cite{venkatakeerthy2020ir2vec}.
% This mapping from LLVM \ac{ir} was also proposed by \citeauthor{ben2018neural}~\cite{ben2018neural}. 
% Their work led to the improved work of \citeauthor{cummins2021programl}~\cite{cummins2021programl}.


% \subsection*{Ithemal Accurate, portable and fast basic block throughput estimation using deep neural networks}\cite{mendis2019ithemal}
% \subsection*{NeuroVectorizer: End-to-End Vectorization with DeepReinforcement Learning}\cite{haj2020neurovectorizer}
% \subsection*{From Loop Fusion to Kernel Fusion: A Domain-Specific Approach to Locality Optimization}\cite{qiao2019loop}
% \subsection*{A Machine Learning Approach for Performance Prediction and Scheduling on Heterogeneous CPUs}\cite{nemirovsky2017machine}




\chapter{Evaluation}
In this chapter we explain how we used the system described in \Cref{sec:approach} to evaluate its performance.
First, we review the results of the \ac{mcts} approach in \Cref{sec:eval:mcts}.
Followingly, we evaluate if we can use the results of the \ac{mcts} approach to learn to generate good schedules with supervised learning methods in \Cref{sec:eval:supervised}.

\section{Hardware selection}
\label{sec:eval:hw}
There are different aspects in choosing the hardware for our experiments.
Two aspects must be given that a processor makes sense.
The first is, that it is supported in LLVM because our whole approach is based on LLVM and we use the LLVM instruction scheduler as a baseline.
Secondly, we limit the hardware choice to processors that implement superscalar pipelines (see \Cref{sec:bg:superscalar-cpu}) because these are used in most modern processors.

However, we consider more aspects regarding the hardware to be important.
The most interesting one is if the superscalar pipeline is implemented as in-order or out-of-order.
While the former comes with no restrictions for our experiments, the latter is able to reschedule instructions during execution in hardware.
Followingly, it is interesting to see how the out-of-order model influences the performance of our approach.

Additionally, to show the versatility of our approach, it is interesting to choose different types of processors.
There exist classical \acp{cpu}, Edge \acp{cpu}, accelerator cards (\eg Graphical Processing Units (GPU), Vector Processing Units, and Intelligence Processing Units (IPU)), and others.

With these aspects in mind we choose to use:
\begin{itemize}
    \item \textbf{Arm Cortex-A53:} This processor is a edge \ac{cpu} and implements an in-order superscalar pipeline based on the AArch64 architecture.
    An edge device that uses it is the Raspberry Pi 3 Model B.
    We use this device for our experiments with a Ubuntu 20.04.
    \item \textbf{\aurora:} This processor is interesting because it implements an out-of-order superscalar pipeline. 
    Additionally, it is an vector processing accelerator card, and thus a different type of processor that is installed via a PCI-Express connection.
\end{itemize}


\section{Approach Validation}
\label{sec:eval:validation}
Before starting to optimize a process, it is useful to validate that there is potential for any optimizations.
Therefore, we show, that different instruction schedules can indeed have different runtimes on our target hardware.
The approach of this experiment is differs from the other experiments because it is an early experiment that took place before our pipeline was developed.
\todo{Is this sentence useful /okay?}

% select longest bb per benchmark
% longest might have the most possible schedule, so more variation in the random schedules
\Cref{sec:approach:dataset} describes the selection of benchmarks from the LLVM Test Suite.
For this experiment we select one basic block per benchmark, for which we modify its instruction schedule.
For the selection of basic blocks we use the heuristic that balances between the most executions and the longest basic blocks, which is discussed in \Cref{sec:approach:basicblock:selection}.
A high number of instructions in a basic block is typically a good indicator for a high number of possible instruction schedules for that basic block. 
A high number of executions ensures that the basic block has a high impact on the runtime of the function that it contains.

% measure the function runtime
% we measure the runtime of the function (implemented llvm passes)
% talk about impact on measurements.
We executed this experiment in a early stage and did not have a basic block extraction pipeline nor means to measure the execution time of a single basic block.
Therefore, we must execute the whole benchmark with the modified instruction schedule of a single basic block.
However, measuring the runtime of the whole benchmark, includes the execution of much overhead code, that we are not interested in.
Thus, we measure the runtime of the function that contains the basic block of interest.
This corresponds to the third method in \Cref{fig:approach:runtime_scopes}.

We implement a pass for the LLVM optimizer, to measure the runtime of a single function.
\todo{Put into approach chapter?}
The pass searches for the function that contains the selected basic block.
Then, it injects calls to the timer functions of the C++ standard library (\lstinline|std::high_resolution_clock::now|).
The calls are injected at the beginning of the given function and right before the return statement.
It injects a compilation-unit-wide global variable, and stores the measurements into this variable.
In the destructor of that compilation-unit, the pass injects code to print all the measurements.

% generate 10 different random schedules 
% do that twice
To generate different different instruction schedules, we choose the simple approach of generating random schedules.
Our random instruction scheduler works on top of a basic list scheduler.
This means, that the list scheduler selects the instructions that are ready for scheduling, and our random scheduler randomly selects one of them.
This is done until no more instructions are left.
We set the seed of the random number generator for reproducibility.

We generate instruction schedules with the seeds 0-10 for each selected basic block, \ie we generate 11 instruction schedules per basic block.
The basic block of interest might execute multiple times in the measured function for reasons discussed in \Cref{sec:approach:runtime-measurement-unit}.
We choose the shortest measured runtime per benchmark run, to ensure that we use the same execution path in our measurements.
To check that the runtime measurements are reproducible, we run the each generated instruction schedule two times.

% evaluate
We have run this experiment on the two processors described in \Cref{sec:eval:hw}.
\Cref{fig:eval:rndm:aarch64} shows a selection of experiment results.
The plots show the runtimes grouped by the different seeds for the random instruction scheduler.
Runtimes that differ between two runs more than 5\% are marked as outliers and plotted in gray.
\Crefrange*{fig:eval:rndm:aarch64:a}{fig:eval:rndm:aarch64:d} show examples where different runtimes are clearly observable for different instruction schedules.
However, this was not always observable.
\Cref{fig:eval:rndm:aarch64:e} and \Cref{fig:eval:rndm:aarch64:f} show examples where no differnce in the runtime was observable.
The average coeffecient of variation over the basic blocks is 0.035.
In summary, we see that different instruction schedules can generate measurable differences in the runtime.
This means that there is potential for improvements.
\begin{figure}
    \begin{subfigure}{0.45\textwidth}
        \includegraphics[width=\textwidth]{img/random-scheduling-experiment-pi-collected/Symbolics-flt-crop.pdf}
        \caption{}
        \label{fig:eval:rndm:aarch64:a}
    \end{subfigure}
    \hfill
    \begin{subfigure}{0.45\textwidth}
        \includegraphics[width=\textwidth]{img/random-scheduling-experiment-pi-collected/trisolv-crop.pdf}
        \caption{}
        \label{fig:eval:rndm:aarch64:b}
    \end{subfigure}
    \begin{subfigure}{0.45\textwidth}
        \includegraphics[width=\textwidth]{img/random-scheduling-experiment-pi-collected/smg2000-crop.pdf}
        \caption{}
        \label{fig:eval:rndm:aarch64:c}
    \end{subfigure}
    \hfill
    \begin{subfigure}{0.45\textwidth}
        \includegraphics[width=\textwidth]{img/random-scheduling-experiment-pi-collected/LoopRerolling-dbl-crop.pdf}
        \caption{}
        \label{fig:eval:rndm:aarch64:d}
    \end{subfigure}
    \begin{subfigure}{0.45\textwidth}
        \includegraphics[width=\textwidth]{img/random-scheduling-experiment-pi-collected/Shootout-matrix-crop.pdf}
        \caption{}
        \label{fig:eval:rndm:aarch64:e}
    \end{subfigure}
    \hfill
    \begin{subfigure}{0.45\textwidth}
        \includegraphics[width=\textwidth]{img/random-scheduling-experiment-pi-collected/fourinarow-crop.pdf}
        \caption{}
        \label{fig:eval:rndm:aarch64:f}
    \end{subfigure}
    \caption[Random Scheduling Experiment on AArch64]{Random Scheduling Experiment on AArch64:
    The bars show the runtime of a function with a random instruction schedule.
    The two runs of the instruction schedule are grouped together.
    Two runs that differ more than 5\% are marked as outliers and plotted in gray.}
    \label{fig:eval:rndm:aarch64}
\end{figure}

\Cref{fig:eval:rndm:aurora} shows a similar selection for the same experiment on the \aurora processor.
We can observe a similar outcome of the experiment.
However, as this processor cannot be interrupted by the \ac{os}, the runtimes are more stable between two runs.
No measurements in the whole example where marked as outliers.
The average coeffecient of variation over the basic blocks is 0.046.
In summary, we observe potential for optimizations on this processor.
\begin{figure}
    \begin{subfigure}{0.45\textwidth}
        \includegraphics[width=\textwidth]{img/random-scheduling-experiment-aurora-collected/Equivalencing-dbl-crop.pdf}
        \caption{}
        \label{fig:eval:rndm:aurora:a}
    \end{subfigure}
    \hfill
    \begin{subfigure}{0.45\textwidth}
        \includegraphics[width=\textwidth]{img/random-scheduling-experiment-aurora-collected/uudecode-crop.pdf}
        \caption{}
        \label{fig:eval:rndm:aurora:b}
    \end{subfigure}
    \begin{subfigure}{0.45\textwidth}
        \includegraphics[width=\textwidth]{img/random-scheduling-experiment-aurora-collected/automotive-susan-crop.pdf}
        \caption{}
        \label{fig:eval:rndm:aurora:c}
    \end{subfigure}
    \hfill
    \begin{subfigure}{0.45\textwidth}
        \includegraphics[width=\textwidth]{img/random-scheduling-experiment-aurora-collected/bicg-crop.pdf}
        \caption{}
        \label{fig:eval:rndm:aurora:d}
    \end{subfigure}
    \begin{subfigure}{0.45\textwidth}
        \includegraphics[width=\textwidth]{img/random-scheduling-experiment-aurora-collected/beamformer-crop.pdf}
        \caption{}
        \label{fig:eval:rndm:aurora:e}
    \end{subfigure}
    \hfill
    \begin{subfigure}{0.45\textwidth}
        \includegraphics[width=\textwidth]{img/random-scheduling-experiment-aurora-collected/uuencode-crop.pdf}
        \caption{}
        \label{fig:eval:rndm:aurora:f}
    \end{subfigure}
    \caption[Random Scheduling Experiment on \aurora]{Random Scheduling Experiment on \aurora:
    The bars show the runtime of a function with a random instruction schedule.
    The two runs of the instruction schedule are grouped together.
    Two runs that differ more than 5\% are marked as outliers.
    However, this processor did not produce any outliers in our experiment.}
    \label{fig:eval:rndm:aurora}
\end{figure}

There are multiple possible reasons that would cause equal measurements in this experiment.
We must differntiate between reasons which mean that different instruction schedules have no effect on the runtime of the basic block and reasons that have its origin in the experiment setup.
We cannot do anything about the former.
Actually, the motivation for this experiment was to verify, that the former reasons do not dominate all the instruction schedules.
There are multiple possibilities for the latter reasons, that have ther origin in the experiment setup:
\begin{itemize}
    \item The basic block for which we manipulate the instruction schedule might have a low influence on the runtime of the function.
        We tried to minimize this effect by choosing basic blocks that are often executed.
    \item Our random instruction scheduler works on top of LLVM.
        LLVM makes, in this stage of the back-end, still use of pseudo instruction that are not represented in the binary.
        This means that schedules that we see as different schedules, might actually not differ in the binary.
    \item There are short functions with a short execution time.
        We observed few changes in the runtime when the measured execution time is below 10,000 processor cycles.
        The underlying timer of the C++ standard library might not be able to measure such short time periods.  
\end{itemize}
However, the experiment is still valid, because we show that we are able to influence the runtime by manipulating the instruction schedules.

In summary, we observe different runtimes for different instruction schedules and the results are reproducible over multiple runs.
This is not true for all basic blocks, but the goal of this experiment was to show the existence of an effect of the instruction schedule on the runtime.
These results motivate the further research on optimizing instruction schedules for these two processors.

\section{MCTS Schedule Search}
\label{sec:eval:mcts}
% Goal of the experiment
We pursue two goals with this experiment.
One is to search instruction schedules that perform better than our baseline.
As our baseline, we choose the LLVM default instruction scheduler, which is defined in the architecture specific compiler back-end.
The second goal is to build a dataset that we can use for our supervised learning approaches.
As a side-effect, we will generate an upper limit for the supervised models.

% What do we do in this experiment
For each basic block under consideration, we first measure the runtime of the basic block compiled with the LLVM default instruction scheduler for the given processor.
Next, we generate an instruction schedule in each \ac{mcts} iteration.
We compile the instruction schedule into an executable format (\Cref{sec:approach:bbisolation}) and execute it to measure the runtime of the basic block of interest (\Cref{sec:approach:datageneration:runtime_methods}).
We train the \ac{mcts} model with the score computed by \Cref{eqn:approach:mcts-score} and start the next iteration to train the \ac{mcts} model.
This way, we generate many instruction schedules per basic block, each evaluated with a score based on their execution time relative to the default instruction schedule.

% Baseline

% How many (and which) BBs do we use
To have a large number of instruction schedules available for our supervised learning methods, we select 20,032 basic blocks from the LLVM Test Suite. \todo{describe how we got the basic blocks}
We generate one \ac{mcts} model for each basic block.
For this experiment, we have selected the longest basic blocks in the dataset.
The number of executions is not relevant in this experiment because we isolate the basic block and measure their specific runtime on the target hardware.
A high number of instructions in the basic blocks helps to avoid trivial scheduling situations.

% Number of steps to outroll the MCTS tree
The experiment is time consuming because of the high number of basic blocks and the expensive compilation and runtime measurements.
We have to terminate the experiment at some point.
After 200 iterations, we have seen that we get a speedup for many basic blocks.
Therefore, we run the \ac{mcts} model for each basic block for 200 iterations.
The experiment execution took 5 weeks for the AArch64, and 3 weeks for the \aurora.

% Exploration vs Exploitation balance weight
% WE DID NOT DESCRIBE THE FORMULAR NOWHERE

% Caching of schedules to detect duplicates
Due to the high cost of the compilation and runtime measurements, we try to avoid these steps as much as possible.
The instruction schedule generation is done in two steps:
We generate the schedule in the LLVM back-end format that can still contain pseudo instructions, and the remaining steps in the LLVM back-end transform this into assembly instructions.
After the removal of pseudo instructions, it can happen that two equal assembly instruction schedules are generated from two different instruction schedules in the LLVM back-end format.
Therefore, we cache the measured runtimes with the hashes of the instruction schedules.
Whenever we already executed a instruction schedule with the same hash, we reuse their measured runtimes.

% Performance: Summary of the excel tables
For the AArch64 we were able to run this experiment on 14,217 basic blocks.
Due to some errors, we measured an unrealistic speed up for some basic blocks.
So, all speed ups greater than a factor of 2 are marked as outliers.
That leaves us with 14,162 valid instruction schedules for the AArch64 processor.
\Cref{tbl:eval:mcts} summarizes the results.
We find better performing instruction schedules for 54.79\% of these basic blocks.
In only 8.24\% of the basic blocks, we did not find an instruction schedule that performed least as good as the LLVM generated one.
On average, we increase the runtime performance of the basic blocks by 8.35\%.
\begin{table}
    \centering
    \begin{tabular}{@{}lrr@{}}
        \toprule
        & \multicolumn{2}{c}{Processor} \\
        \cmidrule{2-3}
        Performance & AArch64 & \aurora \\
        \midrule
        \tblsection{Absolute} && \\
        \tblitem{Better than baseline}    & 54.79\% (7759) & 31.73\% (1349) \\
        \tblitem{Same as baseline}        & 36.97\% (5236) & 53.00\% (2253) \\
        \tblitem{Worse than baseline}     &  8.24\% (1167) & 15.27\%  (649) \\
        \tblsection{Runtime} && \\
        \tblitem{Mean Speed Up} & 8.35\% & 0.30\% \\
        \bottomrule
    \end{tabular}
    \caption[Results of the \ac{mcts} Approach]{Results of the \ac{mcts} approach. The \ac{mcts} approach was very successful on the AArch64 processor. We found better instruction schedules for more than the half of the basic blocks.
    On the \aurora, we are on par with the baseline for half of the basic blocks and fou nd better instruction schedules for a third of the basic blocks.}
    \label{tbl:eval:mcts}
\end{table}

We executed this experiment for 4,253 basic blocks on the \aurora processor.
The lower number of basic blocks is caused by hardware and time limitations.
Only two outliers are generated during this experiment, which results in 4,251 valid basic blocks.
See \Cref{tbl:eval:mcts} for the summarized results.
For this processor, our \ac{mcts} approach found better instruction schedules for 31.73\% of the basic blocks.
In 15.27\% of the basic blocks only worse instruction schedules were found by our model.
The average speed up of the basic blocks is 0.30\%.

% In-order vs OoO discussion (Speed Up vs BB length)
The results could still change in favor of the \aurora processor if we run this experiment on more basic blocks.
However, the result that the performance on this processor is worse than on the AArch64 processor was expected.
The reason is, that the \aurora is an out-of-order processor, and the AArch64 processor is an in-order processor.
Consequently, the \aurora might reschedule the instruction in hardware when it detects problems with the instruction schedule.
Thus, it does not depend on good instruction schedules as the AArch64 processor.

We showed with this experiment that we are able to find better instruction schedules for both our selected processors.
However, our search for instruction schedules was more successful for the AArch64 processor, as does no rescheduling in hardware.

\section{Supervised Schedule Generation}
\label{sec:eval:supervised}
As discussed, the \ac{mcts} approach is not usable for production systems because of its long runtime.
We use another approach for inference here, by using the generated dataset.
First we evaluate the nearest neighbor model, and then the parametric models.

The dataset that we use for our supervised models is based on the results of the \ac{mcts} approach (\Cref{sec:eval:mcts}).
We split this dataset into a training set with 80\% randomly selected data points and a test set with the remaining 20\%.
\Cref{fig:eval:datasets} illustrates the usage of the created datasets.
\begin{figure}
    \centering
    \tikzstyle{defaultnode} = [text centered, align=center, font=\footnotesize\accentfont, rectangle, rounded corners, draw=black, minimum height=1cm, minimum width=2.5cm]
    \tikzstyle{arrow} = [thick,->,>=stealth,-{Latex[scale=1.2]}, font=\footnotesize\accentfont]
    \begin{tikzpicture}
        \node (bb)              [defaultnode] at ( 0,1) {Basic Blocks};
        \node (bbe)             [defaultnode] at ( 4.5,1) {Basic Blocks \\ + \\ Evaluated \\ Instruction Schedules};
        \node (training-set)    [defaultnode] at ( 8.5,2) {Training Set};
        \node (test-set)        [defaultnode] at ( 8.5,0) {Test Set};
        \node (model)           [defaultnode] at (13,2) {Supervised \\ Model};
        \node (eval)            [defaultnode] at (15.5,0) {Supervised \\ Evaluation};

        \draw [arrow] (bb) -- node [midway,above] {MCTS} (bbe);
        \draw [arrow] (bbe) |- node [midway,above] {80\%} (training-set);
        \draw [arrow] (bbe) |- node [midway,below] {20\%} (test-set);
        \draw (training-set) -- node [midway,above=0.4cm] {\footnotesize\accentfont Supervised} (model);
        \draw [arrow] (training-set) -- node [midway,above] {Training} (model);
        \draw [arrow] (test-set) -- node [midway,below] {Model Inference} (eval);
        \draw [arrow] (model) |- (eval);
    \end{tikzpicture}
    \caption[Overview over the used Dataset]{Overview over used datasets. The supervised model is one from \Cref{sec:eval:supervised}.}
    \label{fig:eval:datasets}
\end{figure}

The goal of this experiment is to see if we can generate well performing instruction schedules without auto-tuning methods.

\subsection{Nearest Neighbor Model}
\begin{table}
    \centering
    \begin{tabular}{@{}lrr@{}}
        \toprule
        & \multicolumn{2}{c}{Processor}\\
        \cmidrule{2-3}
        Supervised Model & AArch64 & \aurora \\
        \midrule
        Nearest Neighbor & \textbf{1.38\%} & -3.03\% \\
        \tblsection{Support Vector Regression} && \\
        \tblitem{Balanced + Clustered} & -1.14\% & -3.53\% \\
        \tblitem{Balanced} & -1.18\% & -3.20\% \\
        \tblitem{Clustered} & -1.45\% & \textbf{-2.90\%} \\
        \tblsection{Neural Network} && \\
        \tblitem{Balanced + Clustered} & -0.48\% & -4.19\% \\
        \tblitem{Balanced} & -0.19\% & -3.19\% \\
        \tblitem{Clustered} & -0.47\% & -3.31\% \\
        \bottomrule
    \end{tabular}
    \caption[Performance of our Supervised Models]{Performance of our supervised models relative to the baseline:
    This table shows the mean speedup on the test set with our applied supervised learning models.
    Our nearest neighbor model performed best on the AArch64. 
    It is the only model that generated a positive mean speedup.
    On the \aurora, the SVR model with clustered instructions performed best.
    However, it is still worse than the baseline.}
    \label{tbl:eval:supervised-perf}
\end{table}

We use a big map structure to quickly search our dataset for similar scheduling situations.
The details of this approach are explained in \Cref{sec:app:nearest-neighbor}.
This model is then integrated into the LLVM compiler framework, and we use it to compile our basic blocks in the test set.

The instruction schedules that we compiled with this nearest neighbor model for the AArch64 processor performed better than the baseline instruction scheduler from LLVM.
The measured runtimes for the basic block are 1.38\% shorter.
On the \aurora processor however, the measured runtimes are 3.03\% slower than the basic blocks compiled with the baseline instruction scheduler (see \Cref{tbl:eval:supervised-perf}).

\subsection{Parametric Machine Learning Models}
The parametric models need, an additional data transformation bring it into the form \Cref{eqn:approach:regression-mapping}.
This results in a dataset of 4.9 million data points for the AArch64 architecture, and 1.3 million data points for the \aurora architecture.
We have to added two variations to the parametric approaches which we describe in the next paragraphs.
All approaches are run once with both variations and additionally with only one of the approaches, to see their effect.

% \subsubsection{Instruction Clustering}
There are many similar instructions in the instruction sets of the two processors.
To reduce the dimensionality and simplify the dataset, we cluster some instructions into an alias instruction.
For example, the addition instructions \lstinline|ADDWri| and \lstinline|ADDXri| of the AArch64 architecture are clustered into the same cluster.
These two instructions only differ in that one takes 32-bit values and the other 64-bit values.
See \Cref{app:instr-clusters} for exact clusterings.

% \subsubsection{Dataset Balancing}
Further, we balance our dataset in the target dimension.
The distribution of target values in our dataset follows a normal distribution.
However, this can be problematic because, the model might only learn to predict the mean in any situation.
Therefore, we duplicate samples whose target value is further away from the mean and delete samples whose target value is very close to the mean.
% We sort the dataset into 40 histogram bins.
% In order to not distort the dataset too much, we delete at most 50\% of the samples in a histogram bin and do not increase the number of occurances of a sample to more than 30 times.
\todo{How exactly}
This way we were able to generate a dataset that has a distribution that is closer to an equal distribution.
\begin{figure}
    \centering
    \includegraphics[width=0.75\textwidth]{img/balanced-supervised-dataset-rpi.pdf}
    \caption[Balancing for the AArch64 Dataset]{Balancing for the AArch64 dataset. 
    We duplicate samples with a reward further away from the distribution mean, and delete some samples that are close to the mean.}
    \label{fig:eval:balanced-dataset}
\end{figure}
\Cref{fig:eval:balanced-dataset} illustrates the effect on the distribution of the AArch64-dataset.
The effect for the \aurora dataset is similar.

\subsubsection{Support Vector Regression}
\label{sec:eval:svm}
\acp{svm} have long runtimes when trained with many data points.
We reduced this by randomly selecting 200,000 data points for our model training.

For the AArch64 processor, the average runtime of the instruction schedules generated by this model is between 1.14\% and 1.45\% worse than the runtime of the baseline.
This is the worst result of the parametric models on this processor.

On the \aurora however, we found the best working parametric model to be the \ac{svr} approach combined with clustered instructions.
But it performs still worse than on the AArch64 processor.

Regarding the effects of the dataset balancing and instruction clustering, we see no clear effect.
For the AArch64 processor, the experiments with the balanced datasets perform better.
However, the effect is reversed for the \aurora.
Here, the dataset balancing negatively influenced the performance.

\subsubsection{Neural Network}
\label{sec:eval:nn}
For training the neural network, we use the early stopping scheme.
Once the loss does not improve once for at least $10^{-6}$ in the last 10 epochs, we abort the training.
Therefore, the dataset is split into another training and validation set with a 85/15 split.

The results on the AArch64 processor performs better than the \ac{svr} approach.
With the balanced dataset, we get close to the baseline performance.
However, this approach performs still worse than the baseline.

On the \aurora, the neural network approach performed the worst.
We can also see, that the approach performed the worst with the dataset balancing and the instruction clustering together.


\section{Summary}
The best supervised learning model that we have found is the nearest neighbor approach on the AArch64 processor.
In fact, it is the only one that performed better than the baseline instruction scheduler.
Close to the baseline performance gets the neural network that was trained with the balanced dataset.

For the two dataset variations where we balanced the dataset and clustered similar instructions, we can see that the balancing was helping.
Compare \Cref{tbl:eval:mcts} to see that the models trained with the balanced dataset performed better than with the clustered dataset in three out of four cases.
It even performed better in three out of four cases than the approach with balanced and clustered dataset, and in the one other case it is very close.
So the balancing helped performance wise, and the clustering had a negative influence on the performance.
It seems, that it indeed is important to differenctiate between instructions that only differ in small aspects like the bit width. 

We have seen that our supervised approaches have all worked better on the AArch64 processor.
This was expected due to the smaller dataset and thus the worse average speedup in the \ac{mcts} approach for the \aurora processor.
\Cref{tbl:eval:mcts} shows that the \aurora processor only achieved a 0.30\% speedup in the \ac{mcts} approach.
This value can be seen, as a upper limit for the supervised learning approaches because they are based on the results of the \ac{mcts} approach.

Another important effect, why the approach was not that successful on the \aurora is that it has a out-of-order pipeline.
This means it reschedules the instructions during execution in hardware, so we do not know what is actually executed.
We expected that we would see worse performance on out-of-order hardware.  

Compare speedup with complexity of the problem (number of possible schedulings) vs speedup

Compare CPU Architectures, In-Order vs Out-Of-Order (\url{https://en.wikipedia.org/wiki/Out-of-order_execution})
Here, check speedup vs basic block length, to see if ooo processors perform worse on long basic blocks where it can't see too many instructions ahead

Might be interesting for the discussion: \url{http://www.irisa.fr/alf/downloads/PMA/p241-mcfarlin.pdf}

Mean vs. Median discussion in runtime measurements

\begin{itemize}
    \item Hardware
    \begin{itemize}
        \item Arm Cortex-A53
        \item NEC Aurora
    \end{itemize}
\end{itemize}



Run \ac{mcts} more iterations, because the rest of the schedules still contain many random decision and we can see, that a single decision can make a big difference.


\chapter*{Acronyms}
% The part inside of [], behind \begin{acronym}, must be the longest acronym in the list
\begin{acronym}[MCTS]\itemsep0pt
\acro{ast}[AST]{Abstract Syntax Tree}
\acro{nlp}[NLP]{Natural Language Processing}
\acro{isa}[ISA]{Instruction Set Architecture}
\acro{ir}[IR]{Intermediate Representation}
\acro{xfg}[XFG]{Contextual Flow Graph}
\acro{lstm}[LSTM]{Long short-term memory}
\acro{cpu}[CPU]{central processing unit}
\acro{mcts}[MCTS]{Monte Carlo Tree Search}
\end{acronym}


\printbibliography

\end{document}
