%% This is file `DEMO-TUDaThesis.tex' version 3.07 (2020/10/21),
%% it is part of
%% TUDa-CI -- Corporate Design for TU Darmstadt
%% ----------------------------------------------------------------------------
%%
%%  Copyright (C) 2018--2020 by Marei Peischl <marei@peitex.de>
%%
%% ============================================================================
%% This work may be distributed and/or modified under the
%% conditions of the LaTeX Project Public License, either version 1.3c
%% of this license or (at your option) any later version.
%% The latest version of this license is in
%% http://www.latex-project.org/lppl.txt
%% and version 1.3c or later is part of all distributions of LaTeX
%% version 2008/05/04 or later.
%%
%% This work has the LPPL maintenance status `maintained'.
%%
%% The Current Maintainers of this work are
%%   Marei Peischl <tuda-ci@peitex.de>
%%   Markus Lazanowski <latex@ce.tu-darmstadt.de>
%%
%% The development respository can be found at
%% https://github.com/tudace/tuda_latex_templates
%% Please use the issue tracker for feedback!
%%
%% If you need a compiled version of this document, have a look at
%% http://mirror.ctan.org/tex-archive/macros/latex/contrib/tuda-ci/doc
%% or at the documentation directory of this package (if installed)
%% <path to your LaTeX distribution>/doc/latex/tuda-ci
%% ============================================================================
%%
%%

\documentclass[
	%ngerman,
	ruledheaders=section,%Ebene bis zu der die Überschriften mit Linien abgetrennt werden, vgl. DEMO-TUDaPub
	class=report,% Basisdokumentenklasse. Wählt die Korrespondierende KOMA-Script Klasse
	thesis={type=master},% Dokumententyp Thesis, für Dissertationen siehe die Demo-Datei DEMO-TUDaPhd
	accentcolor=3d,% Auswahl der Akzentfarbe
	% custommargins=true,% Ränder werden mithilfe von typearea automatisch berechnet
	marginpar=false,% Kopfzeile und Fußzeile erstrecken sich nicht über die Randnotizspalte
	%BCOR=5mm,%Bindekorrektur, falls notwendig
	parskip=half-,%Absatzkennzeichnung durch Abstand vgl. KOMA-Sript
	fontsize=11pt,%Basisschriftgröße laut Corporate Design ist mit 9pt häufig zu klein
	logofile=img/tuda_logo, %Falls die Logo Dateien nicht vorliegen
]{tudapub}


% Der folgende Block ist nur bei pdfTeX auf Versionen vor April 2018 notwendig
\usepackage{iftex}
\ifPDFTeX
	\usepackage[utf8]{inputenc}%kompatibilität mit TeX Versionen vor April 2018
\fi

%%%%%%%%%%%%%%%%%%%
%Sprachanpassung & Verbesserte Trennregeln
%%%%%%%%%%%%%%%%%%%
\usepackage[ngerman, main=english]{babel}
\usepackage[autostyle]{csquotes}% Anführungszeichen vereinfacht

% Falls mit pdflatex kompiliert wird, wird microtype automatisch geladen, in diesem Fall muss diese Zeile entfernt werden, und falls weiter Optionen hinzugefügt werden sollen, muss dies über
% \PassOptionsToPackage{Optionen}{microtype}
% vor \documentclass hinzugefügt werden.
\usepackage{microtype}

%%%%%%%%%%%%%%%%%%%
%Literaturverzeichnis
%%%%%%%%%%%%%%%%%%%
\usepackage{biblatex}   % Literaturverzeichnis

%%%%%%%%%%%%%%%%%%%
%Abkürzungsverzeichnis
%%%%%%%%%%%%%%%%%%%
\usepackage[printonlyused, withpage]{acronym}

\usepackage{cleveref}

%%%%%%%%%%%%%%%%%%%
%Paketvorschläge Tabellen
%%%%%%%%%%%%%%%%%%%
%\usepackage{array}     % Basispaket für Tabellenkonfiguration, wird von den folgenden automatisch geladen
\usepackage{tabularx}   % Tabellen, die sich automatisch der Breite anpassen
%\usepackage{longtable} % Mehrseitige Tabellen
%\usepackage{xltabular} % Mehrseitige Tabellen mit anpassarer Breite
\usepackage{booktabs}   % Verbesserte Möglichkeiten für Tabellenlayout über horizontale Linien

%%%%%%%%%%%%%%%%%%%
%Paketvorschläge Mathematik
%%%%%%%%%%%%%%%%%%%
%\usepackage{mathtools} % erweiterte Fassung von amsmath
%\usepackage{amssymb}   % erweiterter Zeichensatz
%\usepackage{siunitx}   % Einheiten

% Diagrams
\usepackage{tikz}
\usetikzlibrary{shapes.geometric, arrows}
\usepackage{epstopdf}

% Code
\usepackage{listings}
\lstset{basicstyle=\ttfamily,breaklines=true}

% Zapf-Dingbats Symbole
\usepackage{pifont}

% TODO Notizen
\usepackage{todonotes}
%\usepackage[disable]{todonotes}
%Formatierungen für Beispiele in diesem Dokument. Im Allgemeinen nicht notwendig!
\let\file\texttt
\let\code\texttt
\let\tbs\textbackslash

% Zapf-Dingbats Symbole
\newcommand*{\FeatureTrue}{\ding{52}}
\newcommand*{\FeatureFalse}{\ding{56}}

\newcommand{\eg}{e.g., }
\newcommand{\ie}{i.e., }

\lstset{
    basicstyle=\ttfamily\small,
    breaklines=true,
    numbers=left,
    stepnumber=1,
    xleftmargin=2.5em,
    frame=lines,
    framexleftmargin=2.5em
}

% ILOC simplifications
\newcommand{\iloc}[1]{\texttt{\small#1}}
\newcommand{\ilocreg}[1]{\iloc{r\textsubscript{#1}}}
\newcommand{\ilocarrow}{$\Rightarrow$}
\newcommand{\iloccmd}[3]{\iloc{#1} & \iloc{#2} & \ilocarrow & \iloc{#3}}
\newcommand{\iloccmdinl}[3]{\iloc{#1}~\iloc{#2}~\ilocarrow~\iloc{#3}}

\newcommand{\aurora}{NEC SX-Aurora TSUBASA-VE}

\newcommand{\tblsection}[1]{\emph{#1}}
\newcommand{\tblitem}[1]{\hspace{0.3cm}#1}

\hyphenation{si-mul-ta-ne-ous-ly}


% Select includes
\includeonly{
	chapters/abstract.tex,
	chapters/introduction,
    chapters/background,
    chapters/related_work,
	chapters/approach,
    chapters/evaluation,
	chapters/conclusion_and_future_work,
	appendices/instruction_clusters
}

\addbibresource{bibliography.bib}

\begin{document}

\Metadata{
	title=Automatic Compiler Customization for Novel Hardware,
	author=Janne Wulf
}

\title{Automatic Compiler Customization for Novel Hardware}
\subtitle{Automatische Anpassung von Compilern für neuartige Hardwarearchitekturen}
\author[J. Wulf]{Janne Wulf}%optionales Argument ist die Signatur,
\birthplace{Oldenburg in Holstein}%Geburtsort, bei Dissertationen zwingend notwendig
\reviewer{Dr. rer. nat. Stefan Guthe \and  Dr.-Ing. Daniel Thuerck}%Gutachter

%Diese Felder erden untereinander auf der Titelseite platziert.
%\department ist eine notwendige Angabe, siehe auch dem Abschnitt `Abweichung von den Vorgaben für die Titelseite'
\department{inf} % Das Kürzel wird automatisch ersetzt und als Studienfach gewählt, siehe Liste der Kürzel im Dokument.
\institute{Interactive Graphics Systems Group}
\group{Perceptual Graphics, Capture and Massively Parallel Computing}

\submissiondate{October 31, 2021}
\examdate{\today}

%	\tuprints{urn=1234,printid=12345,doi=10.25534/tuprints-1234}
%	\dedication{Für alle, die \TeX{} nutzen.}

\maketitle

\affidavit
% oder \affidavit[digital] falls eine rein digitale Abgabe vorgesehen ist.

\chapter*{Abstract}
% What are the goals?
% What are the research questions?
% What did we contribute?
% What are the zielsetzungen?
% What are the results?
With the growing number of processor architectures and specialized accelerators, compilers have become once again a critical tool for effective use of new capabilities.
% Many new compute devices came to the market in the last decade.
Compilers have hardware-dependent parts that need to be manually adjusted and optimized by experts for every target hardware.
One major step of the hardware-dependent compilation process is instruction scheduling.
Regarding their scheduling mechanisms, there are two groups of processors, in-order and out-of-order processors, where the latter can re-schedule instructions during execution in hardware.

This thesis aims to evaluate whether this step can be automatically optimized, especially for novel hardware.
We have multiple contributions through our work:
First, we develop a methodology for automatically creating optimized instruction scheduling policies for any hardware using data-driven approaches.
Next, we build a pipeline for automatic basic block micro-benchmark creation.
A large part of that pipeline is concerned with automatically standardizing basic blocks coming from various C/C++ projects.
Lastly, we collect a dataset of basic blocks extracted from the LLVM Test Suite~\footnote{\url{https://llvm.org/docs/TestSuiteGuide.html}}.

Our experiments aim to compare the effectiveness of our approach on different processors.
We use an in-order AArch64 CPU (ARM Cortex-A53) and the \auroralong{}, an out-of-order vector accelerator.
To summarize, we search for well-performing instruction schedules on a set of micro-benchmarks.
In an effort to reduce inference times, we train different supervised learning models with the results of the search approach.
We generate instruction schedules for the basic blocks in our test dataset, which, on average, perform 8.35\% (in-order) and 0.30\% (out-of-order) better than the LLVM compiler framework.
Our supervised learning models generate instruction schedules that perform, on average, 1.38\% (in-order) better.
On the out-of-order processor, we do not achieve a speedup with the supervised learning models.

\chapter*{Zusammenfassung}
Mit einer steigenden Anzahl an Prozessorarchitekturen und spezialisierten Beschleunigerkarten sind Compiler erneut zu einem entscheidenden Werkzeug für die effektive Nutzung von neuen Fähigkeiten geworden.
Compiler haben hardwarespezifische Teile die für jeden neuen Prozessor von Experten manuell angepasst und optimiert werden müssen.
Ein Hauptschritt des hardwarespezifischen Kompilierens ist das Instruction Scheduling.
Bezüglich ihrer Mechanismen zum Instruction Scheduling können Prozessoren in zwei Gruppen eingeteilt werden: in-order und out-of-order Prozessoren, wobei letztere während der Ausführung Instruktionen in der Hardware noch umsortieren können.

Das Ziel dieser Thesis ist zu untersuchen ob das Instruction Scheduling durch einen automatisierten Vorgang optimiert werden kann, insbesondere für neuartige Hardware.
Wir leisten mehrere Beiträge mit dieser Arbeit:
Wir entwickeln mithilfe von datengetriebenen Ansätzen eine Methodik für die automatisierte Erstellung von optimierten Instruction Scheduling Strategien für jegliche Hardware.
Außerdem erstellen wir eine Folge von Arbeitsschritten für die automatisierte Generierung von Mikrobenchmarks in der Größe von einzelnen Basic Blocks.
Ein großer Teil dieser Arbeitsschritte befasst sich mit der automatischen Vereinheitlichung von Basic Blocks aus verschiedenen C/C++ Projekten.
Zusätzlich erstellen wir einen Datensatz an Basic Blocks aus der LLVM Test Suite~\footnote[1]{\url{https://llvm.org/docs/TestSuiteGuide.html}}.

Das Ziel unserer Experimente ist es die Wirksamkeit unserer Methoden auf verschiedenen Prozessoren zu untersuchen.
Dafür, benutzen wir einen in-order AArch64 Prozessor (ARM Cortex-A53) und die out-of-order Beschleunigerkarte \auroralong{}.
Wir suchen für die Basic Blocks in unserem Datensatz Instruction Schedules, die zu einer kürzeren Laufzeit führen.
Um die langen Laufzeiten des Suchansatzes zu umgehen trainieren wir mit dessen Ergebnissen mehrere Supervised Learning Modelle.
Für die Basic Blocks in unserem Testdatensatz generieren wir so Instruction Schedules die, im Vergleich zu jenen des LLVM Compiler Frameworks, durschnittlich zu einer um 8.35\% (in-order) bzw. 0.30\% (out-of-order) kürzeren Laufzeit führen.
Die Supervised Learning Modelle generieren Instruction Schedules die im Schnitt 1.38\% (in-order) schneller sind.
Für den out-of-order Prozessor erreichen wir keine Beschleunigung mit den Supervised Learning Modellen.


\tableofcontents

% \listoftodos

\chapter{Introduction}

% ACHIEVEMENTS
% Pipeline
% Upper limit for improving the instr scheduling for AARCH64 (~8%) and Aurora
% Scheduling performance improvement for AARCH64 (~1.8%?) and Aurora


% Compare approach with auto-tuning approach.
% Why not use auto-tuning? Search space size, and lack of generalization could be a reason

\chapter{Background}
% NOTE: Assume that the reader has common computer science knowledge!
In this chapter we are introducing knowledge which is required to understand the content of this thesis.
In \cref{sec:bg:compilers} we give an overview about how a compiler is typically implemented and which problems it has to solve that we want to tackle.
The \crefrange{sec:bg:compilers:frontend}{sec:bg:compilers:optimizer} give a superficial overview of the first two phases of a typical compiler.
\Cref{sec:bg:compilers:backend} gives a more detailed introduction to compiler back ends, which we aim to optimize in this thesis.
\todo{explain also the other sections}

%\section{Motivation?}
%\cite{goodman1988code} state the important interdependence between instruction scheduling and register allocation.

\section{CPU Functionality}
ISA\\
Memory and Registers\\
How do CPUs work in general? -> Fetch, Decode, etc.\\
In-order vs out-of-order architectures

\section{Compilers}
\label{sec:bg:compilers}
%What is a compiler? \\
% \todo{This is probably too basic and not required. Might maybe be used in introduction, though}
% In the early days of computer programming, computers were programmed in assembly languages.
% These languages are exclusive to specific processor architectures and only provide the instructions that are available by the \ac{isa} of the processor.
% This approach has several disadvantages.
% Lots of complicated code, that is hard to understand, is written to express even simple programs.
% Also, each program has to be rewritten to execute on a processor with another architecture.
%
% Nowadays computers are programmed---almost exclusively---in high-level programming languages like C/C++, Java, Python, etc.
% The processor is not able to execute code written in such high-level programming languages, though.
% For this reason a compiler has to translate the code into a format that the processor can understand.
Making computer programs, that are written in high-level programming languages~(\eg C/C++, Java, Rust), executable on a specific machine is not a trivial task.
Compilers are only one piece in the tool-chain required to make a program executable.
The compiler translates the high-level language into assembly language, which is translated into object code by the assembler.
Basic functionality like allocating memory or outputting strings on the screen is implemented in a standard library.
The object code of the standard library and potentially other libraries are linked together with the translated program by the linker.
There is more required to execute the code on a specific machine (\eg a runtime library), but explaining this would go beyond the scope of this thesis.

%Why are compilers important? \\
%What are the different tasks a compiler has? \\
The pure translation of the program is only one of the tasks a compiler has to fulfill.
It also has to assure that the program is written in correct syntax of the high-level language.
Most compilers will also optimize the given code and the translated code since a simple one-to-one translation would have a very poor performance.
Eventually there has to be a mapping from variables and to the main memory and the processors registers.
These are all by itself complex problems which are handled by a compiler.

%How are compilers usually implemented? (Front-End, ...)\\
Compilers are usually implemented in different phases to seperate the different tasks and have a structured approach.
A common approach to structure a compiler is by having a front end~(\cref{sec:bg:compilers:frontend}), an optional optimization~(\cref{sec:bg:compilers:optimizer}) and a back end phase~(\cref{sec:bg:compilers:backend}).
These phases are explained in more detail in the following sub sections.

\subsection{Front End}
\label{sec:bg:compilers:frontend}
The front end phase is the first step in the translation process.
Its implementation is dependent on the source language that has to be translated.
A typical front end includes a scanner, a syntax checker/parser, a context-sensitive analysis and translation into a \ac{ir}.
The scanner translates a stream of characters into a stream of tokens that are classified as parts of the source language.
These tokens are then taken by the parser and are checked against the grammer defined by the source language.
Even with a syntactically correct program, there can still be errors in the code, \eg assignments of incompatible types.
These are checked during the context-sensitive analysis phase.
Eventually, the source code is translated into some kind of a \ac{ir} which will be used as input to the optimizer and the back end.

There might be additional steps required depending on the source language.
C/C++ compilers, for example, use a preprocessor to replace macros like \lstinline[language=C]|#include| and \lstinline[language=C]|#define| with their actual values.

\subsection{Optimizer}
\label{sec:bg:compilers:optimizer}
The usage of a \ac{ir}, not only abstracts away the source language and the target hardware, but also permits to apply more passes in the compilation process.
These addition passes transform \ac{ir} to \ac{ir}.
Note that this step is optional and not required to produce correct translations.
The purpose of this step is to optimize the code, in a source and target independent manner, for more efficient execution.
Efficient can mean different things here. % \eg faster, lower memory usage, lower energy consumption.
The transformed code might produce, \eg a faster program, a program that is smaller in size, or a program with less power consumption.
There exist a large amount of optimization passes in most compilers.
They range from rather simple optimizations, like replacing constant variables with their actual value, to more advanced ones that might, for example, simpliy computations with the rules of algebra.

\subsection{Back End}
\label{sec:bg:compilers:backend}

\subsubsection{Instruction Scheduling}
Add Example: e.g. see \url{https://youtu.be/brpomKUynEA?t=271}

\subsubsection{Register Allocation}

\section{LLVM Compiler Infrastructure}
\subsection{Intermediate Representation}
\subsection{Instruction Selection DAG}
\missingfigure[figwidth=\linewidth]{Selection DAG}
\subsection{Pre-RA-Scheduling}
Welche gibt es?\\
Wie funktionieren sie?\\
Welche Infos nutzen sie?\\
\subsection{Post-RA-Scheduling}

\section{Reinforcement Learning}

\chapter{Related Work}
\label{sec:rw}
In this chapter, we survey the existing relevant research in the topic area of this thesis. 
We present the fundamental research and newer data-driven developments in the related fields of instruction scheduling (\Cref{sec:rw:instruction-scheduling}) and register allocation (\Cref{sec:rw:register-allocation}).
Further, we discuss some other relevant works in the fields of machine learning based compiler optimizations, runtime estimation, code feature extraction and machine learning approaches on other scheduling tasks (\Cref{sec:rw:other}).

\section{Instruction Scheduling}
\label{sec:rw:instruction-scheduling}
\subsection{Classical Approaches}
Scheduling problems appear in many fields.
This is why general scheduling is a topic with much existing research.
Also the research on instruction scheduling has a long history.

Algorithms exist that can generate perfect instruction schedules for simple situations with perfect information.
The requirements are met for architectures with only one functional unit and uniform instruction latencies.
The best-known algorithms in this field is the Sethi-Ullman labelling algorithm~\cite{sethi1970generation} and the work by \citeauthor{proebsting1991linear}~\cite{proebsting1991linear}.

However, these conditions are not present in modern processors.
In more complex situations, the instruction scheduling problem is NP-complex~\cite{hennessy1983postpass}.
Modern processors use pipelines to achieve instruction parallelism, see \Cref{sec:bg:cpu}.
Consequently, most instruction schedulers that are used nowadays are based on the list scheduling framework, which was proposed by \citeauthor{landskov1980local}~\cite{landskov1980local}.
The algorithms, that follow this approach, are better able to generate instruction schedules for pipelined processors.
\citeauthor{heller1961sequencing}~\cite{heller1961sequencing} published an early work on how to approach instruction scheduling for these processors.
Much research on further developments of the list scheduling was published~\cite{bernstein1991global,gibbons1986efficient,hennessy1983postpass}.

As elaborated in \Cref{sec:bg:cpu}, the available information on instruction latencies is mostly uncertain.
The reasons are instruction-level parallelism and uncertain memory latencies.
One way to approach this problem, is balanced scheduling~\cite{kerns1993balanced,lo1995improving}.
Another proposed idea, was to use stochastic instruction scheduling~\cite{schielke2000stochastic}.

Instruction scheduling typically works on a basic block level.
This also means that transitions between basic blocks are not scheduled to work well together.
However, research exists on extending the scope to greater regions~\cite{fisher1981trace,bernstein1991global,hwu1993superblock}.

\subsection{Machine Learning Approaches}
The first work that combined data-driven methods with instruction scheduling was a patent by \citeauthor{tarsy1994method}~\cite{tarsy1994method}, filed in \citeyear{tarsy1994method}.
They optimize weights that are used in cost-based heuristics.
These heuristics are used in list scheduling for pipelined processors.

List schedulers usually have multiple heuristics that are used for choosing an instruction from the list of available instructions.
The selection is based on a weighted sum of the heuristics.
\citeauthor{beaty1996using}~\cite{beaty1996using} published a work in \citeyear{beaty1996using}, in which they have used a genetic algorithm to learn weights for different heuristics.
They achieved a 5\% performance increase compared to a random scheduler on three architectures.

\citeauthor{moss1997learning}~\cite{moss1997learning} trained a function that would prefer one instruction over another when presented the previosuly scheduled instructions.
They used decision trees, look-up tables, ELF function approximations, and feed-forward neural networks.
The decision tree performed best and found often the optimal schedule.
However, they only used simulations and limited the basic block length to 10 instructions.

A reinforcement learning and a search heuristic were proposed by \citeauthor{mcgovern1999scheduling}~\cite{mcgovern1999scheduling,mcgovern2002building}.
Their reinforcement learning heuristic sometimes found a better instruction schedule than their baseline.
The long-running search approach found a better instruction schedule every time.
However, their baseline was only a random instruction scheduler and they have only used simulation results.

\citeauthor{russell2006learning}~\cite{russell2006learning} used decision trees to create heuristics to improve instruction scheduling decisions.
They showed, that they generated better instruction schedules 7.8 times more often than the compared heuristics.
Their results are also based on simulations only.

A newer work in this field was published by \citeauthor{jain2019learning}~\cite{jain2019learning}.
They trained a neural network to imitate the instruction schedules by the GCC compiler.
However, this approach is limited by the performance of the GCC instruction scheduler and cannot exceed it.

We conclude, that the machine learning based approaches mostly performed well in theory, but only against weak random baselines.
The only work, that we found that was evaluated on hardware was~\cite{beaty1996using}.

% \citeauthor{cavazos2004inducing} showed that instruction scheduling only makes a difference on some basic blocks.
% They used decision trees for selecting basic blocks for instruction scheduling~\cite{cavazos2004inducing}.

\section{Register Allocation}
\label{sec:rw:register-allocation}
We have disussed the implications of the instruction scheduling phase on the register allocation \Cref{sec:bg:compilers:backend}.
This interdependence was also shown by \citeauthor{goodman1988code}~\cite{goodman1988code}.
\citeauthor{lavrov1962store} showed the connection between the graph-coloring problem and register allocation and thus, the NP-completeness~\cite{lavrov1962store}.
The first graph-coloring based algorithm implemented in a compiler by \citeauthor{chaitin1982register}~\cite{chaitin1982register}.

In the field of register allocation also appeared research that builds the connection to data-driven methods.
\citeauthor{das2019deep} used a deep learning approach to solve the graph coloring problem~\cite{das2019deep}.
The newer and naturally better fitting approach with graph neural networks was used by \citeauthor{lemos2019graph}~\cite{lemos2019graph} to solve the graph coloring problem. 

\section{Other}
\label{sec:rw:other}
\subsection{Compiler Optimizations with Machine Learning}
% \cite{mammadli2020static,haj2020neurovectorizer,huang2019autophase,qiao2019loop}
Machine learning approaches are also applied to optimize other parts of the compilation process.
Deep reinforcement learning was successfully applied to the phase-ordering problem by \citeauthor{mammadli2020static}~\cite{mammadli2020static} and \citeauthor{huang2019autophase}~\cite{huang2019autophase}.
Phase-ordering means to select the compilers optimization passes and define its execution order (see \Cref{sec:bg:compilers:optimizer} for information on the optimization phase).
Deep reinforcement learning was also used by \citeauthor{haj2020neurovectorizer}~\cite{haj2020neurovectorizer} to translate loops into vector processing instructions (SIMD).
\citeauthor{wang2009mapping}~\cite{wang2009mapping} used machine learning to predict the optimal number of threads and the optimal scheduling policy for OpenMP parallelized loops.
For more works, see the surveys~\cite{wang2018machine,ashouri2018survey}.

\subsection{Runtime Estimation}
Various tools for throughput and runtime estimation exist, like Ithemal~\cite{mendis2019ithemal}, llvm-mca\footnote{https://llvm.org/docs/CommandGuide/llvm-mca.html}, and Intel Architecture Code Analyzer (IACA)\footnote{https://software.intel.com/content/www/us/en/develop/articles/intel-architecture-code-analyzer.html}.
However, the listed tools only work with the x86 architecture, which we do not use.
Especially the Ithemal~\cite{mendis2019ithemal} project is interesting as they use a neural network to predict the runtime from the basic block.
That means they learned to extract features from the basic block to predict its runtime.
% This could also be interesting in the instruction scheduling task.

% Code representation learning
\subsection{Feature Extraction from Code}
% \cite{ben2018neural,cummins2021programl,brauckmann2020compiler}
The previously cited works on data-driven machine learning optimizations have or might benefit from research whose goal it is to automatically extract features from code.
A similar approach to the word2vec~\cite{mikolov2013efficient} approach in the \ac{nlp} area was proposed by~\cite{ben2018neural,alon2019code2vec}.
\citeauthor{cummins2021programl} developed a method to extract features from code, that is based on graph structures~\cite{cummins2021programl}.
The work proposed by \citeauthor{brauckmann2020compiler}~\cite{brauckmann2020compiler} works similarly, they also work on graph structures and use graph neural networks to extract features.

\subsection{Other Scheduling Tasks with Machine Learning}
\citeauthor{mao2019learning}~\cite{mao2019learning} have used a deep reinforcement learning approach to schedule data-processing jobs onto compute clusters.
This work is interesting because the jobs have dependencies on each other which are represented in a \ac{dag}, just like the instructions in the instruction scheduling problem.

% \subsection*{Register Allocation for Intel Processor Graphics}\cite{chen2018register}

% \section{Compiler Optimization Phase Ordering}
% %Static neural compiler optimization via deep reinforcement learning~\cite{mammadli2020static}\\
% % - Uses only static information extracted from IR 
% % - IR embedded by using ben2018neural
% % - Use deep q learning
% % - training is executed by running the modified IR and measure the speedup 
% % - reward is defined as ln(T(s_t)/T(s_t+1)) with T being the runtime


% \subsection*{Autophase: Compiler phase-ordering for hls with deep reinforcement learning}\cite{huang2019autophase}

% \section{Code Representation}
% \label{sec:rw:code-representation}
% For making use of data driven techniques in the area of compiler optimization, it is required to somehow extract features from the code to make it accessible for data driven algorithms.
% Older works usually made use of approaches that used hand-tuned features.
% \todo{Maybe add references used in https://chriscummins.cc/u/ed/phd-thesis.pdf (3.3.2.1)}

% Recent works are inspired by the advances in the the field of \ac{nlp}, which are caused by neural networks and continuous distributed vectors (referred to as embeddings) \eg, word2vec~\cite{mikolov2013efficient}. 
% Although, human language is different from codes of programming languages in many aspects, embeddings prove to be useful in code related tasks, too.

% Code inputs may be used directly in a high-level programming language or in an \ac{ir} (\eg, LLVM-IR~\cite{LLVM:CGO04}).
% The advantage of using an \ac{ir} is that it is independent of the source programming language and the target architecture.

%Overviews:
%\begin{itemize}
%    \item ProGraML Paper under Motivation
%    \item https://chriscummins.cc/u/ed/phd-thesis.pdf (3.3.2.1)
%    \item https://arxiv.org/pdf/1904.03061.pdf
%\end{itemize}

% Most approaches for representing high-level language code use some sort of the \ac{ast} in combination with various learning mechanisms.
% %code2vec: Learning distributed representations of code \cite{alon2019code2vec}
% % - works on source code
% % - sensitive to identifier names
% \citeauthor{alon2019code2vec}~\cite{alon2019code2vec} used paths of the \ac{ast} in combination with a Attention Neural Network model.
% Others have used the \ac{ast} in combination with Gated Graph Neural Networks~\cite{ye2020deep, allamanis2017learning}, with Support Vector Machines~\cite{park2012using} or with \ac{lstm} Networks for tree structures~\cite{dam2018deep}.
%\cite{ye2020deep} % Deep Program Structure Modeling Through Multi-RelationalGraph-based Learning, graph-based deep learning (Gated Graph Neural Networks), AST
%\cite{allamanis2017learning} % LEARNING  TO REPRESENT PROGRAMS WITH GRAPHS, graph-based deep learning (Gated Graph Neural Networks), AST
%\cite{dam2018deep} % A deep tree-based model for software defect prediction, AST, tree-LSTM
%\cite{park2012using} % Using Graph-Based ProgramCharacterization for Predictive Modeling, SVM

% \subsubsection{Neural code comprehension: A learnable representation of code semantics~\cite{ben2018neural}}
% - Defines embedding space inst2vec
% - Encodes LLVM-IR, independent of source programming language
% - Leveraging data- and control flow (Contextual Flow Graphs)
% - Use RNN
% - Analyze the embeddings qualitatively using analogies and clustering, and evaluate the learned representation on three different high-level tasks
% - See description in ProGraML Paper under Motivation
% With Neural Code Comprehension (inst2vec)~\cite{ben2018neural}, Ben-Nun et al. defined an embedding space for the LLVM-IR.
% Relevant information to discover code semantics are data and control flow. 
% To emphasize the semantics, the data and control flow are represented in a novel graph structure, called \acp{xfg}.
% %Before building the \ac{xfg}, the LLVM-IR code is split into basic blocks, so diverging control flow is eliminated.
% The context of an individual statement, with size $N$, is defined as the statement and its graph neighbors that are connected by a path of length $N$.
% This statement is then mapped to its embedding by using the skip-gram model~\cite{mikolov2013distributed}, which are known to work good in \ac{nlp} tasks.
% The \ac{xfg} captures features like data and control dependence's, instructions and data types, which are important for our task.

% \subsection*{ProGraML: Graph-based Deep Learning for Program Optimization and Analysis}\cite{cummins2021programl}
%\begin{itemize}
%    \item approach is insensitive to identifier names and preserves operand order and type information
%    \item compared to inst2vec, it can do the same plus preserve operand order, important to distinguish non-commutative ops
%    \item represent programs as directed multigraphs where statements, identifiers, and immediate values are vertices, and relations between vertices are edge
%    \item encode IR into a graph which will be consumed by a Message Passing Neural Network to execute some task
%\end{itemize}

% \subsection*{Compiler-based graph representations for deep learning models of code}\cite{brauckmann2020compiler}
% \todo{Find Paper PDF and write text}

%IR2Vec: A Flow Analysis based Scalable Infrastructure for Program Encodings \cite{keerthy2019ir2vec}
%\begin{itemize}
%    \item Abstracts away the width of the datatype
%\end{itemize}
% IR2Vec~\cite{keerthy2019ir2vec} is another approach that maps an \ac{ir} to a embedding space.
% However, the datatype size, which is important for code optimizations, is abstracted away during the embedding process.

% \section{Applied Machine Learning on Code}
% \label{sec:rw:applied-code-ml}
% Machine learning approaches are used in newer research for code analysis and for compiler optimizations.
% Before machine learning algorithms can be applied, the code must be transformed into some format that the algorithm can consume.

% Various approaches exist that are inspired by the advances in the \ac{nlp} field.
% Much success in the \ac{nlp} field is based on word embeddings~\cite{mikolov2013efficient}.
% Embeddings map words into a high-dimensional continous vector space, such that words with a similar meaning are close to each other.
% \citeauthor{alon2019code2vec}~\cite{alon2019code2vec} used paths of the \ac{ast} in combination with a Attention Neural Network model to map Java code snippets into an embedding space.
% A more broadly applicable approach is to map from instructions of the LLVM \ac{ir} into an embedding space, because it is independent of the source programming language and the target hardware architecture.
% A recent work, that implements this mapping is from \citeauthor{venkatakeerthy2020ir2vec}~\cite{venkatakeerthy2020ir2vec}.
% This mapping from LLVM \ac{ir} was also proposed by \citeauthor{ben2018neural}~\cite{ben2018neural}. 
% Their work led to the improved work of \citeauthor{cummins2021programl}~\cite{cummins2021programl}.


% \subsection*{Ithemal Accurate, portable and fast basic block throughput estimation using deep neural networks}\cite{mendis2019ithemal}
% \subsection*{NeuroVectorizer: End-to-End Vectorization with DeepReinforcement Learning}\cite{haj2020neurovectorizer}
% \subsection*{From Loop Fusion to Kernel Fusion: A Domain-Specific Approach to Locality Optimization}\cite{qiao2019loop}
% \subsection*{A Machine Learning Approach for Performance Prediction and Scheduling on Heterogeneous CPUs}\cite{nemirovsky2017machine}




\chapter{Approach}
\todo{Why are we using a data-driven / machine learning approach? Show that it is complicated to do it by hand. Also write why not to do auto-tuning.}

\begin{figure}
    \centering
    \includegraphics[width=\textwidth]{img/ppt/approach_overview-crop.pdf}
    \caption[Overview of the approach]{Overview of the overall approach. 
    The process covers the selection of basic blocks, the instruction schedule generation, the execution model, the learning process and the derivation of the final scheduler.}
    \label{fig:approach:overview}
\end{figure}
In the rest of this chapter, we will present the approach used for this thesis (compare \cref{fig:approach:overview}).
We start with an overview of the benchmark programs we used (\cref{sec:approach:dataset}).
Then we discuss the importance of basic blocks for instruction scheduling in general and our approach specifically (\cref{sec:approach:basicblock}).
That is followed by explaining the MCTS approach (\cref{subsec:approach:ml:mcts}).
Eventually, we introduce how we combine individual MCTS models into a global model (\cref{subsec:approach:ml:global}) and derive an applicable scheduler(\cref{sec:approach:ml-scheduler}).

\section{Dataset}
\label{sec:approach:dataset}
\begin{enumerate}
    \item 
    \begin{itemize}
        \item What: LLVM test suite
        \item Why: free and ready to use collection of many benchmarks and test programs
        \item How: describe the dataset, add list of benchmarks?, how many benchmarks, how many basic blocks
    \end{itemize}
    \item
    \begin{itemize}
        \item What: Custom compilation process
        \item Why: More control about the compilation process is required. Insertion of optimizer passes, selection of custom schedulers
        \item How: Extract compilation arguments and execution arguments (flags, files to be read) from test suite
    \end{itemize}
\end{enumerate}
    
\section{Basic Block --- scheduling unit}
\label{sec:approach:basicblock}
\begin{itemize}
    \item What: We choose basic blocks as the scheduling unit
    \item Why: Used by the LLVM compiler framework and most other compilers, Limits scope and complexity
    \item How: Code split into BB's and scheduling of BB's independently
\end{itemize}

\subsection{Selection process}
\begin{itemize}
    \item What: Select BB's for our dataset
    \item Why: Too many to work on all of them, time intensive, compilation time, execution time, ML
    \item How: Develop heuristics for the selection
\end{itemize}
\subsubsection{Longest basic blocks}
\begin{itemize}
    \item What: The BB's with the most instructions in them
    \item Why: Good heuristic for number of scheduling decisions. From many scheduling decisions we can learn more. Short BBs have very few scheduling decisions
    \item How: Count the number of instructions in the intermediate LLVM IR files of that basic block
\end{itemize}
\subsubsection{Most executed basic blocks}
\begin{itemize}
    \item What: The BB's that are executed the most often
    \item Why: Most important blocks. Good decisions here can lead to higher speedup
    \item How: Implement LLVM optimizer pass to inject counters into each BB. Execute to count. Generate List of counts
\end{itemize}
\todo{Show example numbers}
\subsubsection{Most executed and longest basic blocks}
\begin{itemize}
    \item What: Combine the two previous heuristics
    \item Why: Most executed, can still be very short BB's. Like initilization of for loops
    \item How: Multiply both numbers
\end{itemize}
\todo{Show example}

\section{Learning to schedule}
\subsection{Local MCTS model}
\label{subsec:approach:ml:mcts}
\subsection{Global model}
\label{subsec:approach:ml:global}

\section{Data generation}
\begin{enumerate}
    \item 
    \begin{itemize}
        \item What: Need to generate data from which we can learn
        \item Why: Dataset has only the code, we need som metric to learn from
        \item How: Execute the code and measure the metric
    \end{itemize}
    \item
    \begin{itemize}
        \item What: We choose to optimize for the runtime
        \item Why: That's what users in the most situations are interested in
        \item How: Execute the BB and measure the runtime
    \end{itemize}
\end{enumerate}

\subsection{Runtime measurement unit}
It would be optimal to just measure the the scheduled instructions with perfect precision and reliability.
\subsubsection{Basic Block}
\begin{itemize}
    \item What: Measuring the basic block itself is complicated
    \item Why: we need very precise time measurements
    \item How: the typical length of a basic block ranges from a hand full of instructions to a few dozens
    \todo{Plot basic block length distribution}
\end{itemize}
\subsubsection{Function}
A specific basic block of a function might be executed multiple times during function execution.
In this situation, this is relevant in to different aspects.

% \begin{itemize}
%     \item What: Advantage: BB is executed multiple times, so its runtime is easier to measure
%     \item Why: 
%     \item How: 
% \end{itemize}
We are more interested in speedups, rather than precise execution times.


\begin{itemize}
    \item What: Measure the function which contains the basic block is bad in general
    \item Why: Different paths could be taken throught the function, e.g. if-else, loops
    \item How: 
\end{itemize}
\begin{itemize}
    \item What: In this case it is okay
    \item Why: the benchmarks are deterministic, each path is taken the same number of times between executions
    \item How: 
\end{itemize}
\subsubsection{Program}
\begin{itemize}
    \item What: Measure the execution time of the whole program
    \item Why: Easy, but unreliable because of IO operations and other noise (e.g. OS), we are only interested in a small fraction of the code
    \item How: 
    \todo{Show visualization}
\end{itemize}

\subsection{Runtime measurement methods}
\todo{Could add little experiment where we compare hardware timers vs OS access}
\subsubsection{Profiling}
\begin{itemize}
    \item What: Profilers are a bad choice for measuring precise runtimes
    \item Why: They are just not designed for it, they serve other purposes.
        Profilers are making snapshots of the running system to measure performance. 
        There is no measurement from point A to point B which leads to inaccuracies
    \item How: Read e.g. the paper of gperf \cite{graham1982gprof}
\end{itemize}
\subsubsection{Operating system methods}
% In linux(C): clock_gettime().
% Windows: QueryPerformanceCounter().
% C++: std::high_resolution_clock().
These might return different clocks depending on the used hardware.
Returns the time from the hardware the OS runs on.
Has overhead but also handles problems.
Typically best for measurements in microseconds range and longer.
\subsubsection{Hardware performance counters}
Accessed by assembly instructions.
% x86: 'rdtsc' Saves current value into EDX:EAX registers
% ARM32: mrc p15m, 0, \%0, c15, c12, 1
% AARCH64: mrs \%0, PMCCNTR_EL0
% Aurora: fencei; smir \%0, \%usrcc
Measurements in CPU cycles.
Lowest overhead available.
Access might be protected (Linux module required).
\subsection{Basic block isolation}
\subsubsection{Basic block extraction}
How to get the assembly (compilation process).
Problems: function calls (remove), jumps(only in last instr, remove), load/access (memory access, readress to stack variable)
\todo{Create table with removed instructions}
\subsubsection{Isolated basic block execution}
From extracted basic block create inline assembly with stack variable. Now executable from C/C++.
Warmup 100 runs. Timings 1000 runs. 
\subsection{Computing rewards from runtimes}
There are some outliers in the runtimes. Sort runtimes, cutoff lower and upper 5\% of the runtimes to remove outliers and compute avg of the rest.
What could be reasons for outliers?
Show some distributions of measurements to justify why we are throwing away data.

\section{Application of the learned model}
\label{sec:approach:ml-scheduler}

% Probably wrong place here, but compare approach with auto-tuning approach.
% Why not use auto-tuning? Search space size, and lack of generalization could be a reason

% \section{Breaking down the problem (maybe just chapter introduction)}
% \begin{itemize}
%     \item Schedulers run on basic blocks -> Select the most executed (hottest) BB's
% \end{itemize}

% \section{Experimentation Pipeline}
% Explain and illustrate the pipeline incl. injection of timers and counters

% \section{Benchmarking}
% \begin{itemize}
%     \item Benchmarking methods: Instrumentation, sampling -> only instrumentation makes sense here. Justify this with instrumentation vs. sampling results (our timing vs. perf timing)
%     \item Where to inject timer in BB DAG
%     \begin{itemize}
%         \item We only want to measure optimized code, but not too kleinteilig because of lacking accuracy. Problematic code is IO code
%         \item Time whole function?
%         \item Time only relevant BB's with surrounding ones (e.g., loop headers) -> Where exactly place the the timer
%         \item Provide data and examples to demonstrate decision making
%     \end{itemize} 
% \end{itemize}
% \subsection{Implementation}
% Injection of the \lstinline[language=C++]{std::chrono::high_resolution_clock}

% \section{Training setup}
% Compile with default scheduler and measure its runtime.
% Compile with our scheduler, execute the program, measure the runtime and use it to train the agent.

% \section{MCTS}
% \subsection{Tree Modeling}
% We probably want to model the interdependence between instructions. 
% But depending on the benchmark programs, only a small fraction of possible edges between instructions are available.

% \eg When in the benchmark a MOV was scheduled and after that, there are only SUB and ADD.
% We cannot schedule another MOV even though that might be the optimal schedule.

% This problem must be addressed somehow.
% There are different possibilities:
% \begin{itemize}
%     \item Declare all the states with their possible successors to different states.
%     The problem with this approach is that the number of states gets very high and most states are visited rarely, possibly only in the given benchmark/basic-block.
%     \item Probablistic Policy like in RL. (Does that exist for MCTS?)
%     This might solve the problem with unavailable edges, but it does not help with learning at all because the agent would have to learn which instructions are available, too.
%     Which is not what we want.
% \end{itemize}
\chapter{Evaluation}
In this chapter we explain how we used the system described in \Cref{sec:approach} to evaluate its performance.
First, we review the results of the \ac{mcts} approach in \Cref{sec:eval:mcts}.
Followingly, we evaluate if we can use the results of the \ac{mcts} approach to learn to generate good schedules with supervised learning methods in \Cref{sec:eval:supervised}.

\section{Hardware selection}
\label{sec:eval:hw}
There are different aspects in choosing the hardware for our experiments.
Two aspects must be given that a processor makes sense.
The first is, that it is supported in LLVM because our whole approach is based on LLVM and we use the LLVM instruction scheduler as a baseline.
Secondly, we limit the hardware choice to processors that implement superscalar pipelines (see \Cref{sec:bg:superscalar-cpu}) because these are used in most modern processors.

However, we consider more aspects regarding the hardware to be important.
The most interesting one is if the superscalar pipeline is implemented as in-order or out-of-order.
While the former comes with no restrictions for our experiments, the latter is able to reschedule instructions during execution in hardware.
Followingly, it is interesting to see how the out-of-order model influences the performance of our approach.

Additionally, to show the versatility of our approach, it is interesting to choose different types of processors.
There exist classical \acp{cpu}, Edge \acp{cpu}, accelerator cards (\eg Graphical Processing Units (GPU), Vector Processing Units, and Intelligence Processing Units (IPU)), and others.

With these aspects in mind we choose to use:
\begin{itemize}
    \item \textbf{Arm Cortex-A53:} This processor is a edge \ac{cpu} and implements an in-order superscalar pipeline based on the AArch64 architecture.
    An edge device that uses it is the Raspberry Pi 3 Model B.
    We use this device for our experiments with a Ubuntu 20.04.
    \item \textbf{\aurora:} This processor is interesting because it implements an out-of-order superscalar pipeline. 
    Additionally, it is an vector processing accelerator card, and thus a different type of processor that is installed via a PCI-Express connection.
\end{itemize}


\section{Approach Validation}
\label{sec:eval:validation}
Before starting to optimize a process, it is useful to validate that there is potential for any optimizations.
Therefore, we show, that different instruction schedules can indeed have different runtimes on our target hardware.
The approach of this experiment is differs from the other experiments because it is an early experiment that took place before our pipeline was developed.
\todo{Is this sentence useful /okay?}

% select longest bb per benchmark
% longest might have the most possible schedule, so more variation in the random schedules
\Cref{sec:approach:dataset} describes the selection of benchmarks from the LLVM Test Suite.
For this experiment we select one basic block per benchmark, for which we modify its instruction schedule.
For the selection of basic blocks we use the heuristic that balances between the most executions and the longest basic blocks, which is discussed in \Cref{sec:approach:basicblock:selection}.
A high number of instructions in a basic block is typically a good indicator for a high number of possible instruction schedules for that basic block. 
A high number of executions ensures that the basic block has a high impact on the runtime of the function that it contains.

% measure the function runtime
% we measure the runtime of the function (implemented llvm passes)
% talk about impact on measurements.
We executed this experiment in a early stage and did not have a basic block extraction pipeline nor means to measure the execution time of a single basic block.
Therefore, we must execute the whole benchmark with the modified instruction schedule of a single basic block.
However, measuring the runtime of the whole benchmark, includes the execution of much overhead code, that we are not interested in.
Thus, we measure the runtime of the function that contains the basic block of interest.
This corresponds to the third method in \Cref{fig:approach:runtime_scopes}.

We implement a pass for the LLVM optimizer, to measure the runtime of a single function.
\todo{Put into approach chapter?}
The pass searches for the function that contains the selected basic block.
Then, it injects calls to the timer functions of the C++ standard library (\lstinline|std::high_resolution_clock::now|).
The calls are injected at the beginning of the given function and right before the return statement.
It injects a compilation-unit-wide global variable, and stores the measurements into this variable.
In the destructor of that compilation-unit, the pass injects code to print all the measurements.

% generate 10 different random schedules 
% do that twice
To generate different different instruction schedules, we choose the simple approach of generating random schedules.
Our random instruction scheduler works on top of a basic list scheduler.
This means, that the list scheduler selects the instructions that are ready for scheduling, and our random scheduler randomly selects one of them.
This is done until no more instructions are left.
We set the seed of the random number generator for reproducibility.

We generate instruction schedules with the seeds 0-10 for each selected basic block, \ie we generate 11 instruction schedules per basic block.
The basic block of interest might execute multiple times in the measured function for reasons discussed in \Cref{sec:approach:runtime-measurement-unit}.
We choose the shortest measured runtime per benchmark run, to ensure that we use the same execution path in our measurements.
To check that the runtime measurements are reproducible, we run the each generated instruction schedule two times.

% evaluate
We have run this experiment on the two processors described in \Cref{sec:eval:hw}.
\Cref{fig:eval:rndm:aarch64} shows a selection of experiment results.
The plots show the runtimes grouped by the different seeds for the random instruction scheduler.
Runtimes that differ between two runs more than 5\% are marked as outliers and plotted in gray.
\Crefrange*{fig:eval:rndm:aarch64:a}{fig:eval:rndm:aarch64:d} show examples where different runtimes are clearly observable for different instruction schedules.
However, this was not always observable.
\Cref{fig:eval:rndm:aarch64:e} and \Cref{fig:eval:rndm:aarch64:f} show examples where no differnce in the runtime was observable.
The average coeffecient of variation over the basic blocks is 0.035.
In summary, we see that different instruction schedules can generate measurable differences in the runtime.
This means that there is potential for improvements.
\begin{figure}
    \begin{subfigure}{0.45\textwidth}
        \includegraphics[width=\textwidth]{img/random-scheduling-experiment-pi-collected/Symbolics-flt-crop.pdf}
        \caption{}
        \label{fig:eval:rndm:aarch64:a}
    \end{subfigure}
    \hfill
    \begin{subfigure}{0.45\textwidth}
        \includegraphics[width=\textwidth]{img/random-scheduling-experiment-pi-collected/trisolv-crop.pdf}
        \caption{}
        \label{fig:eval:rndm:aarch64:b}
    \end{subfigure}
    \begin{subfigure}{0.45\textwidth}
        \includegraphics[width=\textwidth]{img/random-scheduling-experiment-pi-collected/smg2000-crop.pdf}
        \caption{}
        \label{fig:eval:rndm:aarch64:c}
    \end{subfigure}
    \hfill
    \begin{subfigure}{0.45\textwidth}
        \includegraphics[width=\textwidth]{img/random-scheduling-experiment-pi-collected/LoopRerolling-dbl-crop.pdf}
        \caption{}
        \label{fig:eval:rndm:aarch64:d}
    \end{subfigure}
    \begin{subfigure}{0.45\textwidth}
        \includegraphics[width=\textwidth]{img/random-scheduling-experiment-pi-collected/Shootout-matrix-crop.pdf}
        \caption{}
        \label{fig:eval:rndm:aarch64:e}
    \end{subfigure}
    \hfill
    \begin{subfigure}{0.45\textwidth}
        \includegraphics[width=\textwidth]{img/random-scheduling-experiment-pi-collected/fourinarow-crop.pdf}
        \caption{}
        \label{fig:eval:rndm:aarch64:f}
    \end{subfigure}
    \caption[Random Scheduling Experiment on AArch64]{Random Scheduling Experiment on AArch64:
    The bars show the runtime of a function with a random instruction schedule.
    The two runs of the instruction schedule are grouped together.
    Two runs that differ more than 5\% are marked as outliers and plotted in gray.}
    \label{fig:eval:rndm:aarch64}
\end{figure}

\Cref{fig:eval:rndm:aurora} shows a similar selection for the same experiment on the \aurora processor.
We can observe a similar outcome of the experiment.
However, as this processor cannot be interrupted by the \ac{os}, the runtimes are more stable between two runs.
No measurements in the whole example where marked as outliers.
The average coeffecient of variation over the basic blocks is 0.046.
In summary, we observe potential for optimizations on this processor.
\begin{figure}
    \begin{subfigure}{0.45\textwidth}
        \includegraphics[width=\textwidth]{img/random-scheduling-experiment-aurora-collected/Equivalencing-dbl-crop.pdf}
        \caption{}
        \label{fig:eval:rndm:aurora:a}
    \end{subfigure}
    \hfill
    \begin{subfigure}{0.45\textwidth}
        \includegraphics[width=\textwidth]{img/random-scheduling-experiment-aurora-collected/uudecode-crop.pdf}
        \caption{}
        \label{fig:eval:rndm:aurora:b}
    \end{subfigure}
    \begin{subfigure}{0.45\textwidth}
        \includegraphics[width=\textwidth]{img/random-scheduling-experiment-aurora-collected/automotive-susan-crop.pdf}
        \caption{}
        \label{fig:eval:rndm:aurora:c}
    \end{subfigure}
    \hfill
    \begin{subfigure}{0.45\textwidth}
        \includegraphics[width=\textwidth]{img/random-scheduling-experiment-aurora-collected/bicg-crop.pdf}
        \caption{}
        \label{fig:eval:rndm:aurora:d}
    \end{subfigure}
    \begin{subfigure}{0.45\textwidth}
        \includegraphics[width=\textwidth]{img/random-scheduling-experiment-aurora-collected/beamformer-crop.pdf}
        \caption{}
        \label{fig:eval:rndm:aurora:e}
    \end{subfigure}
    \hfill
    \begin{subfigure}{0.45\textwidth}
        \includegraphics[width=\textwidth]{img/random-scheduling-experiment-aurora-collected/uuencode-crop.pdf}
        \caption{}
        \label{fig:eval:rndm:aurora:f}
    \end{subfigure}
    \caption[Random Scheduling Experiment on \aurora]{Random Scheduling Experiment on \aurora:
    The bars show the runtime of a function with a random instruction schedule.
    The two runs of the instruction schedule are grouped together.
    Two runs that differ more than 5\% are marked as outliers.
    However, this processor did not produce any outliers in our experiment.}
    \label{fig:eval:rndm:aurora}
\end{figure}

There are multiple possible reasons that would cause equal measurements in this experiment.
We must differntiate between reasons which mean that different instruction schedules have no effect on the runtime of the basic block and reasons that have its origin in the experiment setup.
We cannot do anything about the former.
Actually, the motivation for this experiment was to verify, that the former reasons do not dominate all the instruction schedules.
There are multiple possibilities for the latter reasons, that have ther origin in the experiment setup:
\begin{itemize}
    \item The basic block for which we manipulate the instruction schedule might have a low influence on the runtime of the function.
        We tried to minimize this effect by choosing basic blocks that are often executed.
    \item Our random instruction scheduler works on top of LLVM.
        LLVM makes, in this stage of the back-end, still use of pseudo instruction that are not represented in the binary.
        This means that schedules that we see as different schedules, might actually not differ in the binary.
    \item There are short functions with a short execution time.
        We observed few changes in the runtime when the measured execution time is below 10,000 processor cycles.
        The underlying timer of the C++ standard library might not be able to measure such short time periods.  
\end{itemize}
However, the experiment is still valid, because we show that we are able to influence the runtime by manipulating the instruction schedules.

In summary, we observe different runtimes for different instruction schedules and the results are reproducible over multiple runs.
This is not true for all basic blocks, but the goal of this experiment was to show the existence of an effect of the instruction schedule on the runtime.
These results motivate the further research on optimizing instruction schedules for these two processors.

\section{MCTS Schedule Search}
\label{sec:eval:mcts}
% Goal of the experiment
We pursue two goals with this experiment.
One is to search instruction schedules that perform better than our baseline.
As our baseline, we choose the LLVM default instruction scheduler, which is defined in the architecture specific compiler back-end.
The second goal is to build a dataset that we can use for our supervised learning approaches.
As a side-effect, we will generate an upper limit for the supervised models.

% What do we do in this experiment
For each basic block under consideration, we first measure the runtime of the basic block compiled with the LLVM default instruction scheduler for the given processor.
Next, we generate an instruction schedule in each \ac{mcts} iteration.
We compile the instruction schedule into an executable format (\Cref{sec:approach:bbisolation}) and execute it to measure the runtime of the basic block of interest (\Cref{sec:approach:datageneration:runtime_methods}).
We train the \ac{mcts} model with the score computed by \Cref{eqn:approach:mcts-score} and start the next iteration to train the \ac{mcts} model.
This way, we generate many instruction schedules per basic block, each evaluated with a score based on their execution time relative to the default instruction schedule.

% Baseline

% How many (and which) BBs do we use
To have a large number of instruction schedules available for our supervised learning methods, we select 20,032 basic blocks from the LLVM Test Suite. \todo{describe how we got the basic blocks}
We generate one \ac{mcts} model for each basic block.
For this experiment, we have selected the longest basic blocks in the dataset.
The number of executions is not relevant in this experiment because we isolate the basic block and measure their specific runtime on the target hardware.
A high number of instructions in the basic blocks helps to avoid trivial scheduling situations.

% Number of steps to outroll the MCTS tree
The experiment is time consuming because of the high number of basic blocks and the expensive compilation and runtime measurements.
We have to terminate the experiment at some point.
After 200 iterations, we have seen that we get a speedup for many basic blocks.
Therefore, we run the \ac{mcts} model for each basic block for 200 iterations.
The experiment execution took 5 weeks for the AArch64, and 3 weeks for the \aurora.

% Exploration vs Exploitation balance weight
% WE DID NOT DESCRIBE THE FORMULAR NOWHERE

% Caching of schedules to detect duplicates
Due to the high cost of the compilation and runtime measurements, we try to avoid these steps as much as possible.
The instruction schedule generation is done in two steps:
We generate the schedule in the LLVM back-end format that can still contain pseudo instructions, and the remaining steps in the LLVM back-end transform this into assembly instructions.
After the removal of pseudo instructions, it can happen that two equal assembly instruction schedules are generated from two different instruction schedules in the LLVM back-end format.
Therefore, we cache the measured runtimes with the hashes of the instruction schedules.
Whenever we already executed a instruction schedule with the same hash, we reuse their measured runtimes.

% Performance: Summary of the excel tables
For the AArch64 we were able to run this experiment on 14,217 basic blocks.
Due to some errors, we measured an unrealistic speed up for some basic blocks.
So, all speed ups greater than a factor of 2 are marked as outliers.
That leaves us with 14,162 valid instruction schedules for the AArch64 processor.
\Cref{tbl:eval:mcts} summarizes the results.
We find better performing instruction schedules for 54.79\% of these basic blocks.
In only 8.24\% of the basic blocks, we did not find an instruction schedule that performed least as good as the LLVM generated one.
On average, we increase the runtime performance of the basic blocks by 8.35\%.
\begin{table}
    \centering
    \begin{tabular}{@{}lrr@{}}
        \toprule
        & \multicolumn{2}{c}{Processor} \\
        \cmidrule{2-3}
        Performance & AArch64 & \aurora \\
        \midrule
        \tblsection{Absolute} && \\
        \tblitem{Better than baseline}    & 54.79\% (7759) & 31.73\% (1349) \\
        \tblitem{Same as baseline}        & 36.97\% (5236) & 53.00\% (2253) \\
        \tblitem{Worse than baseline}     &  8.24\% (1167) & 15.27\%  (649) \\
        \tblsection{Runtime} && \\
        \tblitem{Mean Speed Up} & 8.35\% & 0.30\% \\
        \bottomrule
    \end{tabular}
    \caption[Results of the \ac{mcts} Approach]{Results of the \ac{mcts} approach. The \ac{mcts} approach was very successful on the AArch64 processor. We found better instruction schedules for more than the half of the basic blocks.
    On the \aurora, we are on par with the baseline for half of the basic blocks and fou nd better instruction schedules for a third of the basic blocks.}
    \label{tbl:eval:mcts}
\end{table}

We executed this experiment for 4,253 basic blocks on the \aurora processor.
The lower number of basic blocks is caused by hardware and time limitations.
Only two outliers are generated during this experiment, which results in 4,251 valid basic blocks.
See \Cref{tbl:eval:mcts} for the summarized results.
For this processor, our \ac{mcts} approach found better instruction schedules for 31.73\% of the basic blocks.
In 15.27\% of the basic blocks only worse instruction schedules were found by our model.
The average speed up of the basic blocks is 0.30\%.

% In-order vs OoO discussion (Speed Up vs BB length)
The results could still change in favor of the \aurora processor if we run this experiment on more basic blocks.
However, the result that the performance on this processor is worse than on the AArch64 processor was expected.
The reason is, that the \aurora is an out-of-order processor, and the AArch64 processor is an in-order processor.
Consequently, the \aurora might reschedule the instruction in hardware when it detects problems with the instruction schedule.
Thus, it does not depend on good instruction schedules as the AArch64 processor.

We showed with this experiment that we are able to find better instruction schedules for both our selected processors.
However, our search for instruction schedules was more successful for the AArch64 processor, as does no rescheduling in hardware.

\section{Supervised Schedule Generation}
\label{sec:eval:supervised}
As discussed, the \ac{mcts} approach is not usable for production systems because of its long runtime.
We use another approach for inference here, by using the generated dataset.
First we evaluate the nearest neighbor model, and then the parametric models.

The dataset that we use for our supervised models is based on the results of the \ac{mcts} approach (\Cref{sec:eval:mcts}).
We split this dataset into a training set with 80\% randomly selected data points and a test set with the remaining 20\%.
\Cref{fig:eval:datasets} illustrates the usage of the created datasets.
\begin{figure}
    \centering
    \tikzstyle{defaultnode} = [text centered, align=center, font=\footnotesize\accentfont, rectangle, rounded corners, draw=black, minimum height=1cm, minimum width=2.5cm]
    \tikzstyle{arrow} = [thick,->,>=stealth,-{Latex[scale=1.2]}, font=\footnotesize\accentfont]
    \begin{tikzpicture}
        \node (bb)              [defaultnode] at ( 0,1) {Basic Blocks};
        \node (bbe)             [defaultnode] at ( 4.5,1) {Basic Blocks \\ + \\ Evaluated \\ Instruction Schedules};
        \node (training-set)    [defaultnode] at ( 8.5,2) {Training Set};
        \node (test-set)        [defaultnode] at ( 8.5,0) {Test Set};
        \node (model)           [defaultnode] at (13,2) {Supervised \\ Model};
        \node (eval)            [defaultnode] at (15.5,0) {Supervised \\ Evaluation};

        \draw [arrow] (bb) -- node [midway,above] {MCTS} (bbe);
        \draw [arrow] (bbe) |- node [midway,above] {80\%} (training-set);
        \draw [arrow] (bbe) |- node [midway,below] {20\%} (test-set);
        \draw (training-set) -- node [midway,above=0.4cm] {\footnotesize\accentfont Supervised} (model);
        \draw [arrow] (training-set) -- node [midway,above] {Training} (model);
        \draw [arrow] (test-set) -- node [midway,below] {Model Inference} (eval);
        \draw [arrow] (model) |- (eval);
    \end{tikzpicture}
    \caption[Overview over the used Dataset]{Overview over used datasets. The supervised model is one from \Cref{sec:eval:supervised}.}
    \label{fig:eval:datasets}
\end{figure}

The goal of this experiment is to see if we can generate well performing instruction schedules without auto-tuning methods.

\subsection{Nearest Neighbor Model}
\begin{table}
    \centering
    \begin{tabular}{@{}lrr@{}}
        \toprule
        & \multicolumn{2}{c}{Processor}\\
        \cmidrule{2-3}
        Supervised Model & AArch64 & \aurora \\
        \midrule
        Nearest Neighbor & \textbf{1.38\%} & -3.03\% \\
        \tblsection{Support Vector Regression} && \\
        \tblitem{Balanced + Clustered} & -1.14\% & -3.53\% \\
        \tblitem{Balanced} & -1.18\% & -3.20\% \\
        \tblitem{Clustered} & -1.45\% & \textbf{-2.90\%} \\
        \tblsection{Neural Network} && \\
        \tblitem{Balanced + Clustered} & -0.48\% & -4.19\% \\
        \tblitem{Balanced} & -0.19\% & -3.19\% \\
        \tblitem{Clustered} & -0.47\% & -3.31\% \\
        \bottomrule
    \end{tabular}
    \caption[Performance of our Supervised Models]{Performance of our supervised models relative to the baseline:
    This table shows the mean speedup on the test set with our applied supervised learning models.
    Our nearest neighbor model performed best on the AArch64. 
    It is the only model that generated a positive mean speedup.
    On the \aurora, the SVR model with clustered instructions performed best.
    However, it is still worse than the baseline.}
    \label{tbl:eval:supervised-perf}
\end{table}

We use a big map structure to quickly search our dataset for similar scheduling situations.
The details of this approach are explained in \Cref{sec:app:nearest-neighbor}.
This model is then integrated into the LLVM compiler framework, and we use it to compile our basic blocks in the test set.

The instruction schedules that we compiled with this nearest neighbor model for the AArch64 processor performed better than the baseline instruction scheduler from LLVM.
The measured runtimes for the basic block are 1.38\% shorter.
On the \aurora processor however, the measured runtimes are 3.03\% slower than the basic blocks compiled with the baseline instruction scheduler (see \Cref{tbl:eval:supervised-perf}).

\subsection{Parametric Machine Learning Models}
The parametric models need, an additional data transformation bring it into the form \Cref{eqn:approach:regression-mapping}.
This results in a dataset of 4.9 million data points for the AArch64 architecture, and 1.3 million data points for the \aurora architecture.
We have to added two variations to the parametric approaches which we describe in the next paragraphs.
All approaches are run once with both variations and additionally with only one of the approaches, to see their effect.

% \subsubsection{Instruction Clustering}
There are many similar instructions in the instruction sets of the two processors.
To reduce the dimensionality and simplify the dataset, we cluster some instructions into an alias instruction.
For example, the addition instructions \lstinline|ADDWri| and \lstinline|ADDXri| of the AArch64 architecture are clustered into the same cluster.
These two instructions only differ in that one takes 32-bit values and the other 64-bit values.
See \Cref{app:instr-clusters} for exact clusterings.

% \subsubsection{Dataset Balancing}
Further, we balance our dataset in the target dimension.
The distribution of target values in our dataset follows a normal distribution.
However, this can be problematic because, the model might only learn to predict the mean in any situation.
Therefore, we duplicate samples whose target value is further away from the mean and delete samples whose target value is very close to the mean.
% We sort the dataset into 40 histogram bins.
% In order to not distort the dataset too much, we delete at most 50\% of the samples in a histogram bin and do not increase the number of occurances of a sample to more than 30 times.
\todo{How exactly}
This way we were able to generate a dataset that has a distribution that is closer to an equal distribution.
\begin{figure}
    \centering
    \includegraphics[width=0.75\textwidth]{img/balanced-supervised-dataset-rpi.pdf}
    \caption[Balancing for the AArch64 Dataset]{Balancing for the AArch64 dataset. 
    We duplicate samples with a reward further away from the distribution mean, and delete some samples that are close to the mean.}
    \label{fig:eval:balanced-dataset}
\end{figure}
\Cref{fig:eval:balanced-dataset} illustrates the effect on the distribution of the AArch64-dataset.
The effect for the \aurora dataset is similar.

\subsubsection{Support Vector Regression}
\label{sec:eval:svm}
\acp{svm} have long runtimes when trained with many data points.
We reduced this by randomly selecting 200,000 data points for our model training.

For the AArch64 processor, the average runtime of the instruction schedules generated by this model is between 1.14\% and 1.45\% worse than the runtime of the baseline.
This is the worst result of the parametric models on this processor.

On the \aurora however, we found the best working parametric model to be the \ac{svr} approach combined with clustered instructions.
But it performs still worse than on the AArch64 processor.

Regarding the effects of the dataset balancing and instruction clustering, we see no clear effect.
For the AArch64 processor, the experiments with the balanced datasets perform better.
However, the effect is reversed for the \aurora.
Here, the dataset balancing negatively influenced the performance.

\subsubsection{Neural Network}
\label{sec:eval:nn}
For training the neural network, we use the early stopping scheme.
Once the loss does not improve once for at least $10^{-6}$ in the last 10 epochs, we abort the training.
Therefore, the dataset is split into another training and validation set with a 85/15 split.

The results on the AArch64 processor performs better than the \ac{svr} approach.
With the balanced dataset, we get close to the baseline performance.
However, this approach performs still worse than the baseline.

On the \aurora, the neural network approach performed the worst.
We can also see, that the approach performed the worst with the dataset balancing and the instruction clustering together.


\section{Summary}
The best supervised learning model that we have found is the nearest neighbor approach on the AArch64 processor.
In fact, it is the only one that performed better than the baseline instruction scheduler.
Close to the baseline performance gets the neural network that was trained with the balanced dataset.

For the two dataset variations where we balanced the dataset and clustered similar instructions, we can see that the balancing was helping.
Compare \Cref{tbl:eval:mcts} to see that the models trained with the balanced dataset performed better than with the clustered dataset in three out of four cases.
It even performed better in three out of four cases than the approach with balanced and clustered dataset, and in the one other case it is very close.
So the balancing helped performance wise, and the clustering had a negative influence on the performance.
It seems, that it indeed is important to differenctiate between instructions that only differ in small aspects like the bit width. 

We have seen that our supervised approaches have all worked better on the AArch64 processor.
This was expected due to the smaller dataset and thus the worse average speedup in the \ac{mcts} approach for the \aurora processor.
\Cref{tbl:eval:mcts} shows that the \aurora processor only achieved a 0.30\% speedup in the \ac{mcts} approach.
This value can be seen, as a upper limit for the supervised learning approaches because they are based on the results of the \ac{mcts} approach.

Another important effect, why the approach was not that successful on the \aurora is that it has a out-of-order pipeline.
This means it reschedules the instructions during execution in hardware, so we do not know what is actually executed.
We expected that we would see worse performance on out-of-order hardware.  

Compare speedup with complexity of the problem (number of possible schedulings) vs speedup

Compare CPU Architectures, In-Order vs Out-Of-Order (\url{https://en.wikipedia.org/wiki/Out-of-order_execution})
Here, check speedup vs basic block length, to see if ooo processors perform worse on long basic blocks where it can't see too many instructions ahead

Might be interesting for the discussion: \url{http://www.irisa.fr/alf/downloads/PMA/p241-mcfarlin.pdf}

Mean vs. Median discussion in runtime measurements

\begin{itemize}
    \item Hardware
    \begin{itemize}
        \item Arm Cortex-A53
        \item NEC Aurora
    \end{itemize}
\end{itemize}



Run \ac{mcts} more iterations, because the rest of the schedules still contain many random decision and we can see, that a single decision can make a big difference.

\chapter{Conclusion and Future Work}
\label{sec:conclusion}
\section{Future Work}
Nutze mehr Informationen über die Instructions in schedules

\listoffigures
\listoftables
\chapter*{List of Acronyms}
% The part inside of [], behind \begin{acronym}, must be the longest acronym in the list
\begin{acronym}[MCTS]\itemsep0pt
\acro{ast}[AST]{Abstract Syntax Tree}
\acro{nlp}[NLP]{Natural Language Processing}
\acro{isa}[ISA]{Instruction Set Architecture}
\acro{ir}[IR]{Intermediate Representation}
\acro{xfg}[XFG]{Contextual Flow Graph}
\acro{lstm}[LSTM]{Long short-term memory}
\acro{cpu}[CPU]{central processing unit}
\acro{mcts}[MCTS]{Monte Carlo Tree Search}
\end{acronym}


\todo{Check names of authors}
\printbibliography

\appendix
\crefalias{section}{appendix}
\chapter{Instruction Clusters}
\label{appendix:instr-clusters}
\section{AArch64}
\begin{xltabular}{\textwidth}{lX}
    \toprule
    Cluster & Instructions \\
    \midrule
    \endhead

    \bottomrule
    \endfoot

    ADD & ADDSWri, ADDSWrr, ADDSXri, ADDWri, ADDWrr, ADDWrs, ADDWrx, ADDXri, ADDXrr, ADDXrs, ADDXrx \\
    ADDVv & ADDVv4i32v \\
    ADDv & ADDv16i8, ADDv2i32, ADDv2i64, ADDv4i16, ADDv4i32 \\
    ADJCALLSTACKDOWN & ADJCALLSTACKDOWN \\
    ADJCALLSTACKUP & ADJCALLSTACKUP \\
    ADRP & ADRP \\
    AND & ANDSWri, ANDSWrr, ANDWri, ANDWrr, ANDWrs, ANDXri, ANDXrr \\
    ANDv & ANDv16i8 \\
    ASRV & ASRVWr, ASRVXr \\
    BFM & BFMWri, BFMXri \\
    BIC & BICSWrr, BICSXrr, BICWrr, BICWrs \\
    BICv & BICv8i8 \\
    BSPv & BSPv16i8 \\
    CLZ & CLZWr \\
    CMEQv & CMEQv16i8rz, CMEQv2i32rz, CMEQv4i16rz, CMEQv4i32, CMEQv4i32rz \\
    CMGEv & CMGEv2i32rz, CMGEv2i64 \\
    CMGTv & CMGTv2i64 \\
    CMLTv & CMLTv2i32rz \\
    COPY & COPY \\
    COPY\_TO\_REGCLASS & COPY\_TO\_REGCLASS \\
    CPYi64 & CPYi64 \\
    CSEL & CSELWr, CSELXr \\
    CSINC & CSINCWr, CSINCXr \\
    CSINV & CSINVWr, CSINVXr \\
    CSNEG & CSNEGWr \\
    DUPv & DUPv2i32gpr, DUPv2i64gpr, DUPv2i64lane, DUPv4i32gpr, DUPv4i32lane, DUPv8i16gpr \\
    EOR & EORWri, EORWrr, EORWrs, EORXri, EORXrr \\
    EORv & EORv16i8, EORv8i8 \\
    EXTR & EXTRWrri \\
    EXTRACT\_SUBREG & EXTRACT\_SUBREG \\
    EXTv & EXTv16i8 \\
    FABD64 & FABD64 \\
    FABS & FABSDr, FABSSr \\
    FADD & FADDDrr, FADDSrr \\
    FADDv & FADDv2f64, FADDv4f32 \\
    FCMGTv & FCMGTv2f64, FCMGTv4i32rz \\
    FCMP & FCMPDri, FCMPDrr, FCMPSri, FCMPSrr \\
    FCSEL & FCSELDrrr, FCSELSrrr \\
    FCVTD & FCVTDSr \\
    FCVTLv & FCVTLv2i32, FCVTLv4i32 \\
    FCVTMSUW & FCVTMSUWDr \\
    FCVTMUUW & FCVTMUUWDr \\
    FCVTNv & FCVTNv2i32, FCVTNv4i32 \\
    FCVTS & FCVTSDr \\
    FCVTZSSX & FCVTZSSXSri \\
    FCVTZSUW & FCVTZSUWDr, FCVTZSUWSr \\
    FCVTZSv & FCVTZSv2f64 \\
    FDIV & FDIVDrr, FDIVSrr \\
    FDIVv & FDIVv2f64, FDIVv4f32 \\
    FMAX & FMAXDrr, FMAXSrr \\
    FMAXNM & FMAXNMDrr \\
    FMIN & FMINDrr, FMINSrr \\
    FMINNM & FMINNMDrr \\
    FMOV & FMOVDi, FMOVSi \\
    FMOVD0 & FMOVD0 \\
    FMOVS0 & FMOVS0 \\
    FMOVv & FMOVv2f64\_ns, FMOVv4f32\_ns \\
    FMUL & FMULDrr, FMULSrr \\
    FMULv & FMULv1i32\_indexed, FMULv1i64\_indexed, FMULv2f32, FMULv2f64, FMULv2i32\_indexed, FMULv2i64\_indexed, FMULv4f32 \\
    FNEG & FNEGDr, FNEGSr \\
    FNEGv & FNEGv2f64, FNEGv4f32 \\
    FNMUL & FNMULSrr \\
    FRINTM & FRINTMDr \\
    FSQRT & FSQRTDr \\
    FSUB & FSUBDrr, FSUBSrr \\
    FSUBv & FSUBv2f64, FSUBv4f32 \\
    IMPLICIT\_DEF & IMPLICIT\_DEF \\
    INSERT\_SUBREG & INSERT\_SUBREG \\
    INSv & INSvi32lane \\
    LD1Rv & LD1Rv2d\_POST, LD1Rv2s, LD1Rv2s\_POST, LD1Rv4s \\
    LD2Twov & LD2Twov16b, LD2Twov16b\_POST, LD2Twov2d, LD2Twov2d\_POST, LD2Twov4h, LD2Twov4s \\
    LDRBBpost & LDRBBpost \\
    LDRBBpre & LDRBBpre \\
    LDRBBroW & LDRBBroW \\
    LDRBBroX & LDRBBroX \\
    LDRBBui & LDRBBui \\
    LDRBroX & LDRBroX \\
    LDRDpost & LDRDpost \\
    LDRDpre & LDRDpre \\
    LDRDui & LDRDui \\
    LDRHHpost & LDRHHpost \\
    LDRHHpre & LDRHHpre \\
    LDRHHroW & LDRHHroW \\
    LDRHHroX & LDRHHroX \\
    LDRHHui & LDRHHui \\
    LDRHui & LDRHui \\
    LDRQpost & LDRQpost \\
    LDRQroX & LDRQroX \\
    LDRQui & LDRQui \\
    LDRSBWui & LDRSBWui \\
    LDRSHWpost & LDRSHWpost \\
    LDRSHWui & LDRSHWui \\
    LDRSHXpost & LDRSHXpost \\
    LDRSHXui & LDRSHXui \\
    LDRSHoW & LDRSHWroW, LDRSHXroW \\
    LDRSHoX & LDRSHWroX, LDRSHXroX \\
    LDRSWpost & LDRSWpost \\
    LDRSWui & LDRSWui \\
    LDRSpost & LDRSpost \\
    LDRSpre & LDRSpre \\
    LDRSui & LDRSui \\
    LDRWpost & LDRWpost \\
    LDRWpre & LDRWpre \\
    LDRWui & LDRWui \\
    LDRXpost & LDRXpost \\
    LDRXpre & LDRXpre \\
    LDRXui & LDRXui \\
    LDRoW & LDRDroW, LDRSWroW, LDRSroW, LDRWroW, LDRXroW \\
    LDRoX & LDRDroX, LDRSWroX, LDRSroX, LDRWroX, LDRXroX \\
    LDUR & LDURDi, LDURSWi, LDURSi, LDURWi, LDURXi \\
    LDURBBi & LDURBBi \\
    LDURHHi & LDURHHi \\
    LDURHi & LDURHi \\
    LDURQi & LDURQi \\
    LDURSH & LDURSHWi, LDURSHXi \\
    LSLV & LSLVWr, LSLVXr \\
    LSRV & LSRVWr, LSRVXr \\
    MADD & MADDWrrr, MADDXrrr \\
    MOVID & MOVID \\
    MOVIv & MOVIv16b\_ns, MOVIv2d\_ns, MOVIv2i32, MOVIv2s\_msl, MOVIv4i16, MOVIv4i32, MOVIv4s\_msl, MOVIv8b\_ns, MOVIv8i16 \\
    MOVaddr & MOVaddr \\
    MOVaddrCP & MOVaddrCP \\
    MOVi32imm & MOVi32imm \\
    MOVi64imm & MOVi64imm \\
    MSUB & MSUBWrrr \\
    MULv & MULv2i32, MULv4i32, MULv8i16 \\
    MVNIv & MVNIv4i32, MVNIv4s\_msl \\
    NEGv & NEGv2i32, NEGv4i32 \\
    NOTv & NOTv16i8, NOTv8i8 \\
    ORN & ORNWrr, ORNWrs, ORNXrr \\
    ORR & ORRWri, ORRWrr, ORRWrs, ORRXri, ORRXrr, ORRXrs \\
    ORRv & ORRv16i8, ORRv8i8 \\
    REG\_SEQUENCE & REG\_SEQUENCE \\
    REV & REVWr \\
    REV64v & REV64v2i32, REV64v4i32 \\
    SBFM & SBFMWri, SBFMXri \\
    SCVTFUW & SCVTFUWDri, SCVTFUWSri \\
    SCVTFUX & SCVTFUXDri, SCVTFUXSri \\
    SCVTFv & SCVTFv1i32, SCVTFv1i64, SCVTFv2f64, SCVTFv4f32 \\
    SDIV & SDIVWr, SDIVXr \\
    SHLv & SHLv2i32\_shift, SHLv2i64\_shift, SHLv4i32\_shift \\
    SMADDLrrr & SMADDLrrr \\
    SMAXVv & SMAXVv4i32v \\
    SMAXv & SMAXv4i32 \\
    SMULHrr & SMULHrr \\
    SMULLv & SMULLv4i16\_v4i32 \\
    SSHLLv & SSHLLv2i32\_shift, SSHLLv4i16\_shift, SSHLLv4i32\_shift, SSHLLv8i16\_shift \\
    SSHRv & SSHRv2i32\_shift \\
    STRBBpost & STRBBpost \\
    STRBBpre & STRBBpre \\
    STRBBroW & STRBBroW \\
    STRBBroX & STRBBroX \\
    STRBBui & STRBBui \\
    STRDpost & STRDpost \\
    STRDpre & STRDpre \\
    STRDui & STRDui \\
    STRHHroW & STRHHroW \\
    STRHHroX & STRHHroX \\
    STRHHui & STRHHui \\
    STRQpost & STRQpost \\
    STRQpre & STRQpre \\
    STRQroX & STRQroX \\
    STRQui & STRQui \\
    STRSpost & STRSpost \\
    STRSui & STRSui \\
    STRWpost & STRWpost \\
    STRWpre & STRWpre \\
    STRWui & STRWui \\
    STRXpost & STRXpost \\
    STRXpre & STRXpre \\
    STRXui & STRXui \\
    STRoW & STRDroW, STRWroW, STRXroW \\
    STRoX & STRDroX, STRSroX, STRWroX, STRXroX \\
    STUR & STURDi, STURSi, STURWi, STURXi \\
    STURBBi & STURBBi \\
    STURHHi & STURHHi \\
    STURQi & STURQi \\
    SUB & SUBSWri, SUBSWrr, SUBSWrs, SUBSWrx, SUBSXri, SUBSXrr, SUBSXrs, SUBSXrx, SUBWrr \\
    SUBREG\_TO\_REG & SUBREG\_TO\_REG \\
    SUBv & SUBv2i32, SUBv4i32, SUBv8i16 \\
    UADDLv & UADDLv8i16\_v4i32, UADDLv8i8\_v8i16 \\
    UBFM & UBFMWri, UBFMXri \\
    UCVTFUW & UCVTFUWDri \\
    UCVTFUX & UCVTFUXDri, UCVTFUXSri \\
    UCVTFv & UCVTFv1i32, UCVTFv1i64 \\
    UDIV & UDIVWr, UDIVXr \\
    UMADDLrrr & UMADDLrrr \\
    UMLALv & UMLALv4i16\_v4i32 \\
    UMOVv & UMOVvi32, UMOVvi64 \\
    UMULHrr & UMULHrr \\
    USHLLv & USHLLv16i8\_shift, USHLLv4i16\_shift, USHLLv8i8\_shift \\
    USHLv & USHLv2i32, USHLv4i32 \\
    USHRv & USHRv4i32\_shift, USHRv8i16\_shift \\
    XTNv & XTNv2i32, XTNv4i16, XTNv4i32, XTNv8i8 \\
\end{xltabular}

\newpage
\section{\aurora}
\begin{xltabular}{\textwidth}{lX}
    \toprule
    Cluster & Instructions \\
    \midrule
    \endhead

    \bottomrule
    \endfoot

    ADDSL & ADDSLrr \\
    ADDSWSX & ADDSWSXri, ADDSWSXrm, ADDSWSXrr \\
    ADDSWZX & ADDSWZXrm \\
    ADJCALLSTACKDOWN & ADJCALLSTACKDOWN \\
    ADJCALLSTACKUP & ADJCALLSTACKUP \\
    AND & ANDri, ANDrm, ANDrr \\
    BCFLari\_t & BCFLari\_t \\
    BRCFD & BRCFDir, BRCFDrr \\
    BRCFL & BRCFLir, BRCFLrr \\
    BRCFLa & BRCFLa \\
    BRCFS & BRCFSir, BRCFSrr \\
    BRCFW & BRCFWir, BRCFWrr \\
    BSWP & BSWPri \\
    CMOVD & CMOVDrm, CMOVDrr \\
    CMOVL & CMOVLrm, CMOVLrr \\
    CMOVS & CMOVSrr \\
    CMOVW & CMOVWrm, CMOVWrr \\
    CMPSL & CMPSLrr \\
    CMPSWSX & CMPSWSXrr \\
    CMPUL & CMPULir, CMPULrr \\
    CMPUW & CMPUWir, CMPUWrr \\
    COPY & COPY \\
    COPY\_TO\_REGCLASS & COPY\_TO\_REGCLASS \\
    CVTDL & CVTDLr \\
    CVTDQ & CVTDQr \\
    CVTDS & CVTDSr \\
    CVTDW & CVTDWr \\
    CVTLD & CVTLDr \\
    CVTQD & CVTQDr \\
    CVTSD & CVTSDr \\
    CVTSW & CVTSWr \\
    CVTWDSX & CVTWDSXr \\
    CVTWSSX & CVTWSSXr \\
    DIVSL & DIVSLrr \\
    DIVSWSX & DIVSWSXir, DIVSWSXrm, DIVSWSXrr \\
    DIVUL & DIVULrm, DIVULrr \\
    DIVUW & DIVUWrm, DIVUWrr \\
    EXTRACT\_SUBREG & EXTRACT\_SUBREG \\
    FADDD & FADDDrm, FADDDrr \\
    FADDQ & FADDQrr \\
    FADDS & FADDSrm, FADDSrr \\
    FCMPD & FCMPDrr \\
    FCMPS & FCMPSrr \\
    FDIVD & FDIVDrr \\
    FDIVS & FDIVSrr \\
    FMAXD & FMAXDrr \\
    FMAXS & FMAXSrr \\
    FMIND & FMINDrr \\
    FMINS & FMINSrr \\
    FMULD & FMULDrm, FMULDrr \\
    FMULQ & FMULQrr \\
    FMULS & FMULSrr \\
    FSUBD & FSUBDrr \\
    FSUBQ & FSUBQrr \\
    FSUBS & FSUBSrr \\
    GETSTACKTOP & GETSTACKTOP \\
    IMPLICIT\_DEF & IMPLICIT\_DEF \\
    INSERT\_SUBREG & INSERT\_SUBREG \\
    LD & LDrii, LDrri \\
    LD1BSX & LD1BSXrii, LD1BSXrri \\
    LD1BZX & LD1BZXrii, LD1BZXrri \\
    LD2BSX & LD2BSXrii, LD2BSXrri \\
    LD2BZX & LD2BZXrii, LD2BZXrri \\
    LDLSX & LDLSXrii, LDLSXrri \\
    LDLZX & LDLZXrii, LDLZXrri \\
    LDU & LDUrii, LDUrri \\
    LDZ & LDZr \\
    LEA & LEArii, LEArri \\
    LEASL & LEASLrii, LEASLrri \\
    LEASLz & LEASLzii \\
    LEAz & LEAzii \\
    MAXSL & MAXSLrr \\
    MAXSWSX & MAXSWSXrr \\
    MINSL & MINSLrr \\
    MINSWSX & MINSWSXrr \\
    MULSL & MULSLri, MULSLrm, MULSLrr \\
    MULSWSX & MULSWSXri, MULSWSXrm, MULSWSXrr \\
    NND & NNDrm, NNDrr \\
    OR & ORim, ORri, ORrr \\
    RET & RET \\
    SLAWSX & SLAWSXmr, SLAWSXri, SLAWSXrr \\
    SLL & SLLmr, SLLri, SLLrr \\
    SRAL & SRALri, SRALrr \\
    SRAWSX & SRAWSXri, SRAWSXrr \\
    SRL & SRLri, SRLrr \\
    ST & STrii, STrri \\
    ST1B & ST1Brii, ST1Brri \\
    ST2B & ST2Brii, ST2Brri \\
    STL & STLrii, STLrri \\
    STU & STUrii, STUrri \\
    SUBSL & SUBSLir, SUBSLrr \\
    SUBSWSX & SUBSWSXir, SUBSWSXrr \\
    XOR & XORri, XORrm, XORrr \\    
\end{xltabular}

\end{document}
