\chapter*{Abstract}
% What are the goals?
% What are the research questions?
% What did we contribute?
% What are the zielsetzungen?
% What are the results?
Compilers have hardware-dependent parts that need to be manually adjusted and optimized by experts for every target hardware.
Many new compute devices came to the market in the last decade.
One hardware-dependent step of the compilation process is instruction scheduling.
Processors can be grouped into in-order and out-of-order processors, where the latter can re-schedule itself during execution.
This thesis aims to evaluate if this step can be automatically optimized, especially for novel hardware.
We contribute a methodology to automatically optimize the instruction scheduling task in compilers by using data-driven methods.
Therefore, we run we search well-performing instruction schedules on a set of micro-benchmarks.
With these results, we train different supervised learning models to generate instruction schedules of high quality.
We find instruction schedules for our dataset, which, on average, perform 8.35\% (in-order) and 0.30\% better (out-of-order) than the LLVM compiler framework.
Our supervised learning models generate instruction schedules that perform, on average, 1.38\% (in-order) better.
On out-of-order processors, we are not able to achieve a speedup with the supervised learning models.

\chapter*{Zusammenfassung}
