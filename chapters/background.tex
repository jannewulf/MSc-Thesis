\chapter{Background}
% NOTE: Assume that the reader has common computer science knowledge!
In this chapter we are introducing knowledge which is required to understand the content of this thesis.
In \cref{sec:bg:compilers} we give an overview about how a compiler is typically implemented and which problems it has to solve that we want to tackle.
The \crefrange{sec:bg:compilers:frontend}{sec:bg:compilers:optimizer} give a superficial overview of the first two phases of a typical compiler.
\Cref{sec:bg:compilers:backend} gives a more detailed introduction to compiler back ends, which we aim to optimize in this thesis.
\todo{explain also the other sections}

%\section{Motivation?}
%\cite{goodman1988code} state the important interdependence between instruction scheduling and register allocation.

\section{CPU Functionality}
ISA\\
Memory and Registers\\
How do CPUs work in general? -> Fetch, Decode, etc.\\
In-order vs out-of-order architectures

\section{Compilers}
\label{sec:bg:compilers}
%What is a compiler? \\
% \todo{This is probably too basic and not required. Might maybe be used in introduction, though}
% In the early days of computer programming, computers were programmed in assembly languages.
% These languages are exclusive to specific processor architectures and only provide the instructions that are available by the \ac{isa} of the processor.
% This approach has several disadvantages.
% Lots of complicated code, that is hard to understand, is written to express even simple programs.
% Also, each program has to be rewritten to execute on a processor with another architecture.
%
% Nowadays computers are programmed---almost exclusively---in high-level programming languages like C/C++, Java, Python, etc.
% The processor is not able to execute code written in such high-level programming languages, though.
% For this reason a compiler has to translate the code into a format that the processor can understand.
Making computer programs, that are written in high-level programming languages~(\eg C/C++, Java, Rust), executable on a specific machine is not a trivial task.
Compilers are only one piece in the tool-chain required to make a program executable.
The compiler translates the high-level language into assembly language, which is translated into object code by the assembler.
Basic functionality like allocating memory or outputting strings on the screen is implemented in a standard library.
The object code of the standard library and potentially other libraries are linked together with the translated program by the linker.
There is more required to execute the code on a specific machine (\eg a runtime library), but explaining this would go beyond the scope of this thesis.

%Why are compilers important? \\
%What are the different tasks a compiler has? \\
The pure translation of the program is only one of the tasks a compiler has to fulfill.
It also has to assure that the program is written in correct syntax of the high-level language.
Most compilers will also optimize the given code and the translated code since a simple one-to-one translation would have a very poor performance.
Eventually there has to be a mapping from variables and to the main memory and the processors registers.
These are all by itself complex problems which are handled by a compiler.

%How are compilers usually implemented? (Front-End, ...)\\
Compilers are usually implemented in different phases to seperate the different tasks and have a structured approach.
A common approach to structure a compiler is by having a front end~(\cref{sec:bg:compilers:frontend}), an optional optimization~(\cref{sec:bg:compilers:optimizer}) and a back end phase~(\cref{sec:bg:compilers:backend}).
These phases are explained in more detail in the following sub sections.

\subsection{Front End}
\label{sec:bg:compilers:frontend}
The front end phase is the first step in the translation process.
Its implementation is dependent on the source language that has to be translated.
A typical front end includes a scanner, a syntax checker/parser, a context-sensitive analysis and translation into a \ac{ir}.
The scanner translates a stream of characters into a stream of tokens that are classified as parts of the source language.
These tokens are then taken by the parser and are checked against the grammer defined by the source language.
Even with a syntactically correct program, there can still be errors in the code, \eg assignments of incompatible types.
These are checked during the context-sensitive analysis phase.
Eventually, the source code is translated into some kind of a \ac{ir} which will be used as input to the optimizer and the back end.

There might be additional steps required depending on the source language.
C/C++ compilers, for example, use a preprocessor to replace macros like \lstinline[language=C]|#include| and \lstinline[language=C]|#define| with their actual values.

\subsection{Optimizer}
\label{sec:bg:compilers:optimizer}


\subsection{Back End}
\label{sec:bg:compilers:backend}

\subsubsection{Instruction Scheduling}
Add Example: e.g. see \url{https://youtu.be/brpomKUynEA?t=271}

\subsubsection{Register Allocation}

\section{LLVM Compiler Infrastructure}
\subsection{Intermediate Representation}
\subsection{Instruction Selection DAG}
\missingfigure[figwidth=\linewidth]{Selection DAG}
\subsection{Pre-RA-Scheduling}
Welche gibt es?\\
Wie funktionieren sie?\\
Welche Infos nutzen sie?\\
\subsection{Post-RA-Scheduling}

\section{Reinforcement Learning}
