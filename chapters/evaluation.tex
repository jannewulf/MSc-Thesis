\chapter{Evaluation}
\label{eval:svm}
Potential exploration \rightarrow random scheduling \\
Problem: Most timings are very similar or even the same \\
Reasons:
\begin{itemize}
    \item Many assemblies have equal or similar (e.g., switched add and sub) schedules
    \item Where the schedules differ, there are different schedules
    My impression is that the placement of loads and logic/arithmetic instructions has a big influence on the runtime
    \item The LLVM IR makes heavy use of pseudo instructions.
    This results in differently scheduled LLVM IR, but often produces the same or similar assembly files schedules.
    This means the data is very noisy
\end{itemize}

Compare speedup with complexity of the problem (number of possible schedulings) vs speedup

Compare CPU Architectures, In-Order vs Out-Of-Order (\url{https://en.wikipedia.org/wiki/Out-of-order_execution})

Might be interesting for the discussion: \url{http://www.irisa.fr/alf/downloads/PMA/p241-mcfarlin.pdf}

Mean vs. Median discussion in runtime measurements

\section{Hardware}
\subsection{Arm Cortex-A53}
\subsection{NEC Aurora}



Run \ac{mcts} more iterations, because the rest of the schedules still contain many random decision and we can see, that a single decision can make a big difference.
