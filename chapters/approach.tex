\chapter{Approach}
Probably wrong place here, but compare approach with auto-tuning approach.
Why not use auto-tuning? Search space size could be a reason

\section{Breaking down the problem (maybe just chapter introduction)}
\begin{itemize}
    \item Schedulers run on basic blocks -> Select the most executed (hottest) BB's
\end{itemize}

\section{Experimentation Pipeline}
Explain and illustrate the pipeline incl. injection of timers and counters

\section{Benchmarking}
\begin{itemize}
    \item Benchmarking methods: Instrumentation, sampling -> only instrumentation makes sense here. Justify this with instrumentation vs. sampling results (our timing vs. perf timing)
    \item Where to inject timer in BB DAG
    \begin{itemize}
        \item We only want to measure optimized code, but not too kleinteilig because of lacking accuracy. Problematic code is IO code
        \item Time whole function?
        \item Time only relevant BB's with surrounding ones (e.g., loop headers) -> Where exactly place the the timer
        \item Provide data and examples to demonstrate decision making
    \end{itemize} 
\end{itemize}
\subsection{Implementation}
Injection of the \lstinline[language=C++]|std::chrono::high_resolution_clock|
