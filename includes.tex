% Der folgende Block ist nur bei pdfTeX auf Versionen vor April 2018 notwendig
\usepackage{iftex}
\ifPDFTeX
	\usepackage[utf8]{inputenc}%kompatibilität mit TeX Versionen vor April 2018
\fi

%%%%%%%%%%%%%%%%%%%
%Sprachanpassung & Verbesserte Trennregeln
%%%%%%%%%%%%%%%%%%%
\usepackage[ngerman, main=english]{babel}
\usepackage[autostyle]{csquotes}% Anführungszeichen vereinfacht

% Falls mit pdflatex kompiliert wird, wird microtype automatisch geladen, in diesem Fall muss diese Zeile entfernt werden, und falls weiter Optionen hinzugefügt werden sollen, muss dies über
% \PassOptionsToPackage{Optionen}{microtype}
% vor \documentclass hinzugefügt werden.
\usepackage{microtype}

%%%%%%%%%%%%%%%%%%%
%Literaturverzeichnis
%%%%%%%%%%%%%%%%%%%
\usepackage{biblatex}   % Literaturverzeichnis

%%%%%%%%%%%%%%%%%%%
%Abkürzungsverzeichnis
%%%%%%%%%%%%%%%%%%%
\usepackage[printonlyused]{acronym}

\usepackage{cleveref}

%%%%%%%%%%%%%%%%%%%
%Paketvorschläge Tabellen
%%%%%%%%%%%%%%%%%%%
%\usepackage{array}     % Basispaket für Tabellenkonfiguration, wird von den folgenden automatisch geladen
\usepackage{tabularx}   % Tabellen, die sich automatisch der Breite anpassen
%\usepackage{longtable} % Mehrseitige Tabellen
%\usepackage{xltabular} % Mehrseitige Tabellen mit anpassarer Breite
\usepackage{booktabs}   % Verbesserte Möglichkeiten für Tabellenlayout über horizontale Linien

%%%%%%%%%%%%%%%%%%%
%Paketvorschläge Mathematik
%%%%%%%%%%%%%%%%%%%
%\usepackage{mathtools} % erweiterte Fassung von amsmath
%\usepackage{amssymb}   % erweiterter Zeichensatz
%\usepackage{siunitx}   % Einheiten

% Diagrams
\usepackage{tikz}
\usetikzlibrary{shapes.geometric, arrows}
\usepackage{epstopdf}

% Code
\usepackage{listings}
\lstset{basicstyle=\ttfamily,breaklines=true}

% Zapf-Dingbats Symbole
\usepackage{pifont}

% TODO Notizen
\usepackage{todonotes}
%\usepackage[disable]{todonotes}